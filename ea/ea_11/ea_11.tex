\documentclass[a4paper,11pt]{article}
\usepackage[textwidth=170mm, textheight=230mm, inner=20mm, top=10mm, bottom=20mm]{geometry}
\usepackage[normalem]{ulem}
\usepackage[utf8]{inputenc}
\usepackage[T1]{fontenc}
\usepackage{physics}
\PassOptionsToPackage{defaults=hu-min}{magyar.ldf}
\usepackage[magyar]{babel}
\usepackage{amsmath, amsthm,amssymb,paralist,array, ellipsis, graphicx, float}

\begin{document}
\def\R{\mathbb{R}}
\def\biz{\normalsize{\underline{Bizonyítás:} }\hspace*{0.5cm}}
\def\tetel{\normalsize \textbf{\underline{Tétel}: }}
\def\defi{\normalsize \textbf{Definíció: }}
\def\bizva{\quad\blacksquare}
\def\pl{\textbf{Pl:}}
\def\fab{F[a,b]}
\def\fabr{f:(a,b)\to\R}
\def\cab{\in C[a,b]}
\def\rab{R[a,b]}
\def\te{\tau_1}
\def\tk{\tau_2}
\def\ftau{(f,\tau)}
\def\sumi{\sum\limits_{i=1}^{n}}
\def\intv{[x_{i-1},x_i]}
\def\intab{\int\limits_{a}^{b}}
\begin{center}
	{\LARGE\textbf{Analízis 2.}}\\[0.2cm]
	
	{\Large 11. Előadás jegyzet}\\[0.5cm]	
\end{center}
{\small A jegyzetet \textsc{Bauer Bence} készítette \textsc{Dr. Weisz Ferenc} előadása alapján.}\\[0.4cm]
\tetel (Műveletek integrálokkal)\\[0.1cm]
Tfh. $f,g\in\rab$ Ekkor:\\[0.2cm]
\textbf{i,} $f+g\in\rab$ és $\intab f+g=\intab f+\intab g$\\[0.2cm]
\textbf{ii,} $\lambda\cdot f\in\rab$ és $\intab\lambda f=\lambda\cdot\intab f
\quad\lambda\in\R$\\[0.2cm]
\textbf{iii,} $f\cdot g\in\rab$\\[0.2cm]
\textbf{iv,} Ha $\abs{g(x)}\geq m>0\quad\forall x\in[a,b]$, akkor $\frac{f}{g}\in\rab$\\[0.2cm]
\biz Legyen $\tau=\{x_0,x_1,...,x_n\}\in\fab,\\[0.2cm]F_i:=\sup\limits_{\intv}f,
\quad f_i:=\inf\limits_{\intv}f\quad G_i:=\sup\limits_{\intv}g\quad
g_i:=\inf\limits_{\intv}g$\\[0.3cm]
\textbf{i,}$\quad f_i+g_i\leq f(x)+g(x)\leq F_i+G_i,\quad x\in\intv\\[0.2cm]
\Rightarrow f_i+g_i\leq\inf\limits_{\intv}(f+g)\leq\sup\limits_{\intv}(f+g)\leq
F_i+G_i\quad\quad/\cdot(x_i-x_{i-1})\\[0.2cm]\Rightarrow
s\ftau+s(g,\tau)\leq s(f+g,\tau)\leq S(f+g,\tau)\leq S\ftau+S(g,\tau)$\\[0.2cm]
Legyen $\te,\tk\in\fab$ tetszőleges és $\tau=\te\cup\tk\\[0.2cm]\Rightarrow
s(f,\te)+s(g,\tk)\leq s\ftau+s(g,\tau)\leq s(f+g,\tau)\leq I_*(f+g)\leq I^*
(f+g)\leq S(f+g,\tau)\leq\\[0.2cm]\leq S\ftau+S(g,\tau)\leq S(f,\te)+S(g,\tk)
\quad/\cdot\sup\limits_{\te},\inf\limits_{\te},\sup\limits_{\tk},\inf\limits_{\tk}
\\[0.2cm]\Rightarrow I_*(f)+I_*(g)\leq I_*(f+g)\leq I^*(f+g)\leq I^*(f)+I^*(g),
\quad\text{Mivel }I_*(f)=I^*(f)$ (ugyanez $g$-re)\\[0.2cm]
$\Rightarrow I_*(f+g)=I^*(f+g)\text{ és }\intab f+g=\intab f+\intab g$\\[0.2cm]
\textbf{ii,} Tfh. $\lambda\geq0\Rightarrow s(\lambda f,\tau)=\lambda\cdot s\ftau
\quad\quad(\inf\limits_{\intv}\lambda f=\lambda\cdot\inf\limits_{\intv}f)\\[0.2cm]
\Rightarrow I_*(\lambda f)=\lambda\cdot I_*(f)$\hspace{1cm}Hasonlóan:
$S(\lambda f,\tau)=\lambda\cdot S\ftau\Rightarrow I^*(\lambda f)=
\lambda\cdot I^*(f)\\[0.1cm]\Rightarrow I_*(\lambda f)=I^*(\lambda f)\text{ és }
\intab\lambda f=\lambda\cdot\intab f$\\[0.1cm]Tfh. $\lambda<0\\[0.1cm]
s(\lambda f,\tau)=\lambda\cdot S\ftau\Rightarrow I_*(\lambda f)=\lambda\cdot I^*(f)
\quad\text{és}\quad S(\lambda f,\tau)=\lambda\cdot s\ftau\Rightarrow I^*(\lambda f)=
\lambda\cdot I_*(f)\\[0.2cm]\Rightarrow I_*(\lambda f)=I^*(\lambda f)\text{ és }
\intab\lambda f=\lambda\cdot\intab f$\\[0.2cm]\textbf{iii,}
Oszcillációs összeggel: Tfh. $f,g\geq0\quad[a,b]$-n\\[0.1cm]
$f_i\cdot g_i\leq f(x)\cdot g(x)\leq F_i\cdot G_i\quad x\in\intv\\[0.2cm]
\Rightarrow f_i\cdot g_i\leq\inf\limits_{\intv}(f\cdot g)\leq
\sup\limits_{\intv}(f\cdot g)\leq F_i\cdot G_i\\[0.1cm]\Omega(f\cdot g,\tau)=
\sumi(\sup\limits_{\intv}(f\cdot g)-\inf\limits_{\intv}(f\cdot g))\cdot
(x_i-x_{i-1})\leq\sumi(F_i\cdot G_i-f_i\cdot g_i)\cdot(x_i-x_{i-1})=\\[0.1cm]=
\sumi(F_i\cdot G_i-F_i\cdot g_i+F_i\cdot g_i-f_i\cdot g_i)\cdot(x_i-x_{i-1})=
\\[0.1cm]=\sumi F_i(G_i-g_i)\cdot(x_i-x_{i-1})+\sumi g_i(F_i-f_i)\cdot(x_i-x_{i-1})
\\[0.2cm]f\in\rab\Rightarrow f$ korlátos $\Rightarrow F_i\leq M$ és $g_i\leq M\quad
\forall i=1,...,n$\newpage
$\Rightarrow\Omega(f\cdot g,\tau)\leq M\cdot\Omega(g,\tau)+M\cdot\Omega\ftau\\[0.2cm]
\Rightarrow\forall\varepsilon>0,\exists\te:\Omega(g,\te)<\varepsilon\quad\text{és}
\quad\forall\varepsilon>0,\exists\tk:\Omega(f,\tk)<\varepsilon$\\[0.2cm]Legyen
$\tau=\te\cup\tk\Rightarrow\Omega(g,\tau)\leq\Omega(g,\te)<\varepsilon\quad$
Hasonlóan: $\Omega\ftau\leq\Omega(f,\tk)<\varepsilon\\[0.2cm]\Rightarrow
\Omega(f\cdot g,\tau)<2\varepsilon M\Rightarrow f\cdot g\in\rab$\\[0.2cm]
Ha $f$ és $g$ tetszőleges, akkor legyen $m_f:=\inf\limits_{[a,b]}f,\quad
m_g:=\inf\limits_{[a,b]}g\Rightarrow\underbrace{f-m_f}_{\in\rab}\geq0,
\quad\underbrace{g-m_g}_{\in\rab}\geq0\\[0.2cm]\Rightarrow\underbrace{(f-m_f)(g-m_g)}
_{\in\rab}=f\cdot g\underbrace{-g\cdot m_f-f\cdot m_g+m_f\cdot m_g}_{\in\rab}
\Rightarrow f\cdot g\in\rab$\\[0.2cm]\textbf{iv,} Elég: $\frac{1}{g}\in\rab$\\[0.2cm]
\[\frac{1}{g(x)}-\frac{1}{g(y)}=\frac{g(y)-g(x)}{g(x)\cdot g(y)}\leq
\frac{\abs{g(y)-g(x)}}{\abs{g(x)\cdot g(y)}}\leq\frac{G_i-g_i}{m^2}\Rightarrow
\sup\limits_{\intv}\frac{1}{g}-\inf\limits_{\intv}\frac{1}{g}\leq
\frac{G_i-g_i}{m^2}\]
$\Rightarrow\Omega(\frac{1}{g},\tau)=\sumi(\sup\limits_{\intv}\frac{1}{g}-
\inf\limits_{\intv}\frac{1}{g})\cdot(x_i-x_{i-1})\leq
\frac{1}{m^2}\Omega(g,\tau)\\[0.2cm]\forall\varepsilon>0,\exists\tau,\Omega(g,\tau)
<\varepsilon\Rightarrow\Omega(\frac{1}{g},\tau)\leq\frac{\varepsilon}{m^2}\bizva$
\\[0.3cm]\tetel Legyen $c\in[a,b]$, ekkor:\\[0.2cm]
$f\in\rab\Leftrightarrow f\in R[a,c]\text{ és }f\in R[c,b]$ Ekkor:\hspace{0.5cm}
$\intab f=\int\limits_{a}^{c}f+\int\limits_{c}^{b}f$\\[0.2cm]
\underline{Bizonyítás nélkül}\\[0.2cm]
\defi $\intab f=0,\quad a>b:\intab f=-\intab f$\\[0.2cm]
\tetel $f\in R[A,B],\quad a,b,c\in[A,B]$ Ekkor: $\intab f=\int\limits_a^cf+
\int\limits_c^bf$\\[0.2cm]
\tetel Ha $f\cab$, ekkor $f\in\rab$\\[0.2cm]
\biz Ha $f\cab\Rightarrow$ Heine tétel miatt $f$ egyenletesen folytonos, azaz\\[0.2cm]
$\forall\varepsilon>0,\exists\delta>0,\forall x,y\in[a,b],\abs{x-y}<\delta:
\abs{f(x)-f(y)}<\varepsilon$\\[0.2cm]
Legyen $\tau\in\fab$ olyan, hogy $\abs{\tau}<\delta\\[0.2cm]
\Omega\ftau=\sumi(\sup\limits_{\intv}f-\inf\limits_{\intv}f)(x_i-x_{i-1})=\sumi
\underbrace{\sup\limits_{x,y\in\intv}\abs{f(x)-f(y)}}_{<\varepsilon}\cdot(x_i-x_{i-1})
\leq\varepsilon\cdot(b-a)\\[0.2cm]\Rightarrow f\in\rab\bizva$\\[0.2cm]
\tetel $f:[a,b]\to\R$ monoton, ekkor $f\in\rab$\\[0.2cm]
\biz Hasonlóan Tfh. $f\nearrow$\\[0.2cm]$\Omega\ftau=\sumi
(\sup\limits_{\intv}f-\inf\limits_{\intv}f)\cdot(x_i-x_{i-1})=\sumi
(f(x_i)-f(x_{i-1}))\cdot(x_i-x_{i-1})$\\[0.2cm]Tfh. $\abs{\tau}<\delta\Rightarrow
\Omega\ftau\leq\delta\cdot\sumi(f(x_i)-f(x_{i-1}))=\delta\cdot(f(b)-f(a))<\varepsilon$
\\[0.2cm]Ha a $\delta<\frac{\varepsilon}{f(b)-f(a)}\Rightarrow f\in\rab\bizva$
\\[0.2cm]\tetel $f$ értékeit véges sok pontban megváltoztatom $(\tilde{f})$,\\[0.2cm]ha
$f\in\rab$, akkor $\tilde{f}\in\rab\text{ és }\intab\tilde{f}=\intab f$\newpage
\defi $f:[a,b]\to\R$ szakaszonként folytonos, ha $\exists\tau=\{x_0,x_1,...,x_n\}
\in\fab$, hogy\\[0.2cm]$f\in C(x_{i-1},x_i)\text{ és }\exists\lim\limits_{x_i+0}f,
\exists\lim\limits_{x_i-0}f$ és végesek $i=1,...,n$\\[0.2cm]
\tetel Ha $f:[a,b]\to\R$ szakaszonként folytonos, akkor $f\in\rab$ és
$\intab f=\sumi\int\limits_{x_{i-1}}^{x_i}f$\\[0.2cm]\biz
$f\in C(x_{i-1},x_i)\Rightarrow f\in R\intv\bizva$\\[0.3cm]
\tetel $f,g\in\rab$\\[0.2cm]
\textbf{i,} Ha $f\geq0$, akkor $\intab f\geq0$\\[0.2cm]
\textbf{ii,} Ha $f\geq g$, akkor $\intab f\geq\intab g$\\[0.2cm]
\biz \textbf{i,} $f\geq0\Rightarrow s\ftau\geq0\Rightarrow I_*(f)=\intab f\geq0$
\\[0.2cm]\textbf{ii,} $f-g\geq0\Rightarrow\intab(f-g)\geq0\bizva$\\[0.2cm]
\tetel Ha $f\in\rab$, akkor $\abs{f}\in\rab$ és $\quad-\abs{\int f}\leq\abs{\int f}
\leq\int\abs{f}$\\[0.2cm]
\biz $\Omega(\abs{f},\tau)=\sumi(\sup\limits_{\intv}\abs{f}-
\inf\limits_{\intv}\abs{f})\cdot(x_i-x_{i-1})=\\[0.2cm]=\sumi\sup\limits_{x,y\in\intv}
\abs{\abs{f(y)}-\abs{f(x)}}\cdot(x_i-x_{i-1})\leq\sup\limits_{x,y\in\intv}
\abs{f(y)-f(x)}\cdot(x_i-x_{i-1})=\Omega\ftau<\varepsilon\\[0.4cm]\Rightarrow
\abs{f}\in\rab\bizva$
\end{document}