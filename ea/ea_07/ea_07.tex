\documentclass[a4paper,11pt]{article}
\usepackage[textwidth=170mm, textheight=230mm, inner=20mm, top=10mm, bottom=20mm]{geometry}
\usepackage[normalem]{ulem}
\usepackage[utf8]{inputenc}
\usepackage[T1]{fontenc}
\usepackage{physics}
\PassOptionsToPackage{defaults=hu-min}{magyar.ldf}
\usepackage[magyar]{babel}
\usepackage{amsmath, amsthm,amssymb,paralist,array, ellipsis, graphicx, float}

\begin{document}
\def\N{\mathbb{N}}
\def\Z{\mathbb{Z}}
\def\R{\mathbb{R}}
\def\biz{\normalsize{\underline{Bizonyítás:} }\hspace*{0.5cm}}
\def\tetel{\normalsize \textbf{\underline{Tétel}: }}
\def\defi{\normalsize \textbf{Definíció: }}
\def\pl{\textbf{Pl:}}
\def\rtr{\R\to\R}
\def\D{\mathcal{D}}
\def\cab{\in C[a,b]}
\def\dab{\in\D(a,b)}
\def\bizva{\quad\blacksquare}
\def\fda{f\in\D(a)}
\def\limaj{\displaystyle\lim_{a+0}}
\def\limnj{\displaystyle\lim_{0+0}}
\def\Rv{\overline{\mathbb{R}}}
\begin{center}
	{\LARGE\textbf{Analízis 2.}}\\[0.2cm]
	
	{\Large 7. Előadás jegyzet}\\[0.5cm]	
\end{center}
{\small A jegyzetet \textsc{Bauer Bence} készítette \textsc{Dr. Weisz Ferenc} előadása alapján.}\\[0.4cm]
{\large \textbf{\underline{Határértékek}}}\\[0.2cm]
\textbf{Kritikus esetek: ($\frac{0}{0},\frac{\pm\infty}{\pm\infty},0^0,
1^\infty, \infty-\infty$)}\\[0.2cm]\tetel (L'Hospital szabály $\frac{0}{0}$ 
alakra)\\[0.2cm] Tfh. \textbf{i,} $f,g\dab,\quad(-\infty\leq a<b<\infty)$
\\[0.2cm]\hspace*{0.7cm} \textbf{ii,} $g'(x)\neq0,\quad x\in(a,b)$
\\[0.2cm]\hspace*{0.8cm}\textbf{iii,} $\limaj f=\limaj g=0$
\\[0.2cm]\hspace*{0.8cm}\textbf{iv,} $\exists\limaj \frac{f'}{g'}$ és $\limaj
\frac{f'}{g'}=A\in\Rv$\\[0.2cm] Ekkor: $\exists\limaj \frac{f}{g}$ és 
$\limaj\frac{f}{g}=\limaj\frac{f'}{g'}$\\[0.2cm]\biz\textbf{i,} Tfh. 
$a\neq-\infty$\\[0.2cm]Tudjuk: $\limaj\frac{f'}{g'}=A\Rightarrow\forall 
\varepsilon>0,\exists x_0\in(a,b),\forall\xi\in(a,x_0):
\frac{f'(\xi)}{g'(\xi)}\in K_{\varepsilon}(A)$\\[0.2cm]Legyen $f(a)=g(a)=0$ és
legyen $x\in(a,x_0)$ tetszőleges, ekkor $f,g\in C[a,x]$ és $f,g\in\D(a,x)$
\\[0.2cm]$\Rightarrow$ a Cauchy-középértéktétel miatt: $\exists\xi\in(a,x)
:\frac{f(x)-f(a)}{g(x)-g(a)}= \frac{f'(\xi)}{g'(\xi)}\in K_{\varepsilon}(A)
\\[0.2cm]\Rightarrow\forall\varepsilon >0,\exists x_0\in(a,b),\forall 
x\in(a,x_0):\frac{f(x)}{g(x)}\in K_{\varepsilon}(A) \Rightarrow\limaj
\frac{f}{g}=A$\\[0.2cm]\textbf{ii,} Tfh. $a=-\infty\quad$ Visszavezetjük 
\textbf{i,}-re\\[0.1cm]Legyen $F(y):=f(b+1-\frac{1}{y}),\quad y\in(0,1)$ és
\\[0.1cm]$G(y):=g(b+1-\frac{1}{y}),\quad y\in(0,1)$\\[0.1cm]$y<1 \Rightarrow 
b+1-\frac{1}{y}<b\Rightarrow f\text{ és }g$ értelmezve van a 
($b+1-\frac{1}{y}$) pontban.\\[0.1cm]$\limnj F=\lim\limits_{y\to0+0}
f(b+1-\frac{1}{y})= \lim\limits_{-\infty} f=0\\[0.2cm]\limnj G=
\lim\limits_{-\infty}g=0\quad$ Ha $\exists\limnj\frac{F}{G}$, ekkor\\[0.2cm]
$\limnj\frac{F}{G}=\lim\limits_{y\to0+0}\frac{f}{g}(b+1-\frac{1}{y})=
\lim\limits_{-\infty}\frac{f}{g}\\[0.2cm]F'(y)=f'(b+1-\frac{1}{y})\cdot
\frac{1}{y^2}\\[0.2cm]G'(y)=g'(b+1-\frac{1}{y})\cdot\frac{1}{y^2}\neq0
\quad\quad y\in(0,1)\\[0.2cm]\limnj\frac{F'}{G'}=\lim\limits_{-\infty}
\frac{f'}{g'}\hspace{1.5cm}$ Alkalmazható \textbf{i,} $F$ és $G$-re\\[0.2cm]
$\Rightarrow\limnj\frac{F}{G}=\limnj\frac{F'}{G'}\quad\Rightarrow\quad
\lim\limits_{-\infty}\frac{f}{g}=\limnj\frac{F}{G}\text{ és }\lim\limits_{-\infty}\frac{f'}{g'}=\limnj\frac{F'}{G'}\bizva$
\newpage\tetel (L'Hospital szabály $\frac{\infty}{\infty}$ alakra)\\[0.2cm] Tfh. 
\textbf{i,} $f,g\dab,\quad(-\infty\leq a<b<\infty)$\\[0.2cm] 
\hspace*{0.8cm}\textbf{ii,} $g'(x)\neq0,\quad x\in(a,b)$\\[0.2cm] 
\hspace*{0.8cm}\textbf{iii,} $\limaj f=\limaj g=\infty$\\[0.2cm]
\hspace*{0.8cm}\textbf{iv,} $\exists\limaj\frac{f'}{g'}$ és $\limaj
\frac{f'}{g'}=A\in\Rv$\\[0.2cm] Ekkor: $\exists\limaj\frac{f}{g}$ és $\limaj 
\frac{f}{g}=\limaj\frac{f'}{g'}$\\[0.2cm]\biz\textbf{i,} Tfh. $a\neq
-\infty,A\in\R$\\[0.2cm]Tudjuk: $\limaj\frac{f'}{g'}=A\RightarrowÍ\forall
\varepsilon>0,\exists x_0\in(a,b),\forall\xi\in(a,x_0):\frac{f'(\xi)}
{g'(\xi)}\in K_{\varepsilon}(A)$\\[0.1cm]Legyen $x\in(a,x_0)$ és alkalmazzuk a 
Cauchy középérték-tételt az $[x,x_0]$ intervallumra\\[0.1cm] $\Rightarrow
\exists\xi\in(x,x_0):\frac{f(x)-f(x_0)}{g(x)-g(x_0)}=\frac{f'(\xi)}{g'(\xi)}$
\hspace{1cm} Felthető, hogy $f>0\quad(a,x_0)$-n, hiszen $\lim\limits_a 
f=\infty$\\[0.1cm]Hasonlóan $g>0\quad(a,x_0)$-n.\\[0.2cm]$\frac{f(x)}{g(x)}\cdot
\frac{1-\frac{f(x_0)}{f(x)}}{1-\frac{g(x_0)}{g(x)}}=\frac{f'(\xi)}{g'(\xi)}
\Rightarrow\frac{f(x)}{g(x)}=\frac{f'(\xi)}{g'(\xi)}\cdot
\underbrace{\frac{1-\frac{g(x_0)}{g(x)}}{1-\frac{f(x_0)}{f(x)}}}_{T(x)}=
\frac{f'(\xi)}{g'(\xi)}\cdot T(x)=\frac{f'(\xi)}{g'(\xi)}\cdot(T(x)-1)+
\frac{f'(\xi)}{g'(\xi)}\\[0.2cm]\limaj T=1\Rightarrow\limaj(T-1)=0\\[0.2cm]
\frac{f'(\xi)}{g'(\xi)}\in K_{\varepsilon}(A)\quad\Rightarrow\quad A-
\varepsilon<\frac{f'(\xi)}{g'(\xi)}<A+\varepsilon\quad\Rightarrow\quad
\frac{f'(\xi)}{g'(\xi)}$ korlátos.\\[0.2cm]$\Rightarrow\limaj
\frac{f'(\xi)}{g'(\xi)}\cdot(T(x)-1)=0\Rightarrow\forall\varepsilon>0,\exists
x_1\in(a,x_0),\forall x\in(a,x_1):\abs{\frac{f'(\xi)}{g'(\xi)}\cdot(T(x)-1)}<
\varepsilon\\[0.2cm]\frac{f(x)}{g(x)}-A=\frac{f'(\xi)}{g'(\xi)}\cdot(T(x)-1)
+\frac{f'(\xi)}{g'(\xi)}-A\\[0.2cm]\Rightarrow\abs{\frac{f(x)}{g(x)}-A}\leq
\underbrace{\abs{\frac{f'(\xi)}{g'(\xi)}\cdot(T(x)-1)}}_{<\varepsilon}+
\underbrace{\abs{\frac{f'(\xi)}{g'(\xi)}-A}}_{<\varepsilon}<2\varepsilon
\\[0.2cm]\Rightarrow\forall\varepsilon>0,\exists x_1\in(a,x_0),\forall
x\in(a,x_1):\abs{\frac{f(x)}{g(x)}-A}<2\varepsilon\Rightarrow\limaj
\frac{f}{g}=A$\\[0.2cm]\textbf{ii,} $a\neq-\infty,A=\infty\quad$ Láttuk:\\[0.1cm]
$\frac{f(x)}{g(x)}=\frac{f'(x)}{g'(x)}\cdot T(x)\\[0.2cm]\limaj T=1
\Rightarrow\exists x_1\in(a,x_0),\forall x\in(a,x_1):T(x)>\frac{1}{2}
\\[0.2cm]\frac{f'(x)}{g'(x)}\in K_{\varepsilon}(\infty)\Rightarrow
\frac{f'(x)}{g'(x)}>\frac{1}{\varepsilon}\Rightarrow\frac{f(x)}{g(x)}>
\frac{1}{2\varepsilon}\\[0.2cm]\forall\varepsilon>0,\exists x_1\in(a,x_0),
\forall x\in(a,x_1):\frac{f(x)}{g(x)}>\frac{1}{2\varepsilon}\quad\Rightarrow
\quad\limaj\frac{f}{g}=\infty=A$\\[0.2cm]
\textbf{iii,} $a\neq-\infty,A=-\infty\quad$ Hasonló \textbf{ii,}-hez\\[0.2cm]
\textbf{iv,} $a=-\infty$ Visszavezetjük az előzőre mint az előző tétel 
\textbf{ii,} részében. $\bizva$\newpage
\underline{Megj:} \textbf{i,} A tétel igaz baloldali és mindkét oldali
határértékre is.\\[0.2cm]\hspace*{1.6cm}\textbf{ii,} Lehet, hogy többször kell
alkalmazni.\\[0.2cm]\hspace*{1.6cm}\textbf{iii,} A többi kritikus eset 
visszavezethető erre a két esetre.\\[0.2cm]\hspace*{1cm}\pl$\quad0\cdot\infty,
\quad f\cdot g=\frac{f}{\frac{1}{g}}$\\[0.1cm] Alkalmazásaikra példák 
\\[0.2cm]\pl\textbf{ i,} $\lim\limits_{x\to0}\frac{\sin x}{x}=
\lim\limits_{x\to0}\frac{\cos x}{1}=1$\\[0.2cm]\textbf{ii,} $\lim\limits_{x\to0}
\frac{chx-\cos x}{x^2}=\lim\limits_{x\to0}\frac{shx+\sin x}{2x}=
\lim\limits_{x\to0}\frac{ch x+\cos x}{2}=1$\\[0.2cm]\textbf{iii,} 
$\lim\limits_{x\to0+0}x^n\cdot\ln x=\lim\limits_{x\to0+0}\frac{\ln x}{x^{-n}} 
=\lim\limits_{x\to0+0}\frac{x^{-1}}{-n\cdot x^{-n-1}}=\lim\limits_{x\to0+0}- 
\frac{x^n}{n}=0$\\[0.2cm]\textbf{iv,} $\lim\limits_{x\to0+0}x^x= 
\lim\limits_{x\to0+0}e^{\ln x^x}=\lim\limits_{x\to0+0}e^{x\cdot\ln x}=1$
\\[0.4cm]{\large \textbf{\underline{Taylor-sorok}}}\\[0.2cm]
\hspace*{1cm}\underline{Eml:} Tfh. a $\sum\alpha_n(x-a)^n$ hatványsor 
konvergenciasugara $R>0$ és legyen\\[0.2cm]$f(x):=\sum\limits_{n=0}^{\infty} 
\alpha_n(x-a)^n,\quad x\in K_R(a),\quad$ Ekkor: $f\in\D^{\infty}(x)$ és 
\\[0.2cm]$f^{(k)}(x)=\sum\limits_{n=k}^{\infty}\alpha_n\cdot n\cdot(n-1)
\cdot...\cdot(n-k+1)\cdot(x-a)^{n-k}\quad(x\in K_R(a))\quad\text{és}
\quad\alpha_k=\frac{f^{(k)}(a)}{k!}$\\[0.2cm]
\defi \textbf{i,} Ha $f\in\D^{\infty}(a)$, ekkor a $\sum\frac{f^{(n)}(a)}{n!} 
\cdot(x-a)^{n}$ sort, az $f$ függyvény Taylor sorának nevezzük.\\[0.2cm]
\textbf{ii,} Ha $f\in\D^{(n)}(a)$, akkor $\sum\limits_{k=0}^{n} 
\frac{f^{(k)}(a)}{k!}(x-a)^k$ az $f$-nek $n$-edik Taylor polinomja\hspace{1cm}Jel: 
$T_nf(x)$\\[0.2cm]
\underline{Problémák:}\\[0.2cm]\textbf{i,} Konvergens-e a Taylor sor?\\[0.2cm] 
Ha igen, az összeg $=f$-el?\\[0.2cm]Állítás: Ha $f$-nek $\exists$ hatványsora, 
akkor ez a Taylor sor is.\\[0.2cm]
\pl \textbf{i,} $f(x)=\frac{1}{1+x},\quad x>-1,\quad a=0$\\[0.2cm]$f(0)=1, \quad 
f'(x)=-(1+x)^{-2},\quad f'(0)=-1\\[0.2cm]f''(x)=2(1+x)^{-3},\quad 
f''(0)=2\\[0.2cm]f^{(n)}(x)=(-1)^n\cdot n!\cdot(1+x)^{-n-1},\quad 
f^{(n)}(0)=(-1)^n\cdot n!$\\[0.2cm]
A Taylor sor: $\sum\limits_{n=0}\frac{f^{(n)}(0)}{n!}\cdot x^n= 
\sum\limits_{n=0}(-1)^n\cdot x^n=\sum\limits_{n=0}(-x)^n$\\[0.2cm] ez konvergens 
$\Leftrightarrow\abs{x}<1,\quad$ ekkor $\quad\sum\limits_{n=0}^{\infty}
(-x)^n=\frac{1}{1+x}\quad\abs{x}<1$
\end{document}