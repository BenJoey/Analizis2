\documentclass[a4paper,11pt]{article}
\usepackage[textwidth=170mm, textheight=230mm, inner=20mm, top=10mm, bottom=20mm]{geometry}
\usepackage[normalem]{ulem}
\usepackage[utf8]{inputenc}
\usepackage[T1]{fontenc}
\PassOptionsToPackage{defaults=hu-min}{magyar.ldf}
\usepackage[magyar]{babel}
\usepackage{amsmath, amsthm,amssymb,paralist,array, ellipsis, graphicx, float}

\begin{document}
\def\Z{\mathbb{Z}}
\def\R{\mathbb{R}}
\def\limn{\displaystyle\lim_{n\to +\infty}}
\def\limh{\displaystyle\lim_{h\to0}}
\def\limxa{\displaystyle\lim_{x\to a}}
\def\rtr{\displaystyle\R\to\R}
\def\D{\displaystyle\mathcal{D}}
\def\lima{\displaystyle\lim_{a}}
\def\fda{f\in\D(a)}
\begin{enumerate}
	\item \textbf{Mi a belső pont definíciója?}\\[0.1cm]$a\in A\subset\R$ az $A$ belső pontja, ha $\exists K(a)\subset A$
	\item \textbf{Mikor mondja azt, hogy egy $f\in\rtr$ függvény differenciálható valamely pontban?}\\[0.1cm]Az $f\in\rtr$ deriválható (differenciálható) az $a\in int D_f$ pontban, ha $\exists$ és véges a\\[0.1cm]$\limh\frac{f(a+h)-f(a)}{h}=f'(a)$ határérték.
	\item \textbf{Mi a kapcsolat a pontbeli differenciálhatóság és a folytonosság között?}\\[0.1cm]$f\in\rtr,\quad a\in int\D_f$, ekkor\hspace{1cm} $\fda\Rightarrow f\in C(a)$
	\item \textbf{Mi a jobb oldali derivált definíciója?}\\[0.1cm]Legyen $f\in\rtr,a\in\D_f\\[0.1cm]\displaystyle\lim_{x\to a+0}\frac{f(x)-f(a)} {x-a}=:f'_+(a),\quad$ az $f$ jobboldali deriváltja az $a$-ban.
	\item \textbf{Milyen ekvivalens átfogalmazást ismer a pontbeli deriválhatóságra a lineáris \\ közelítéssel?}\\[0.1cm]$f\in\rtr,\quad a\in int\D_f\quad$ Ekkor\\[0.1cm]$\fda\Leftrightarrow\exists A\in\R, \exists\epsilon:\D_f\to\R,\lima\epsilon=0$ és $f(x)-f(a)=A(x-a)+ \epsilon(x)(x-a)\quad(x\in\D_f)$\\[0.1cm]Ekkor $A=f'(a)$
\end{enumerate}
\end{document}