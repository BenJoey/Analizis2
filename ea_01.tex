\documentclass[a4paper,11pt]{article}
\usepackage[textwidth=170mm, textheight=230mm, inner=20mm, top=10mm, bottom=20mm]{geometry}
\usepackage[normalem]{ulem}
\usepackage[utf8]{inputenc}
\usepackage[T1]{fontenc}
\PassOptionsToPackage{defaults=hu-min}{magyar.ldf}
\usepackage[magyar]{babel}
\usepackage{amsmath, amsthm,amssymb,paralist,array, ellipsis, graphicx, float}

\usepackage{hyperref}
\hypersetup{
	colorlinks = true	
}
\begin{document}
\def\biz{\normalsize{\textbf{\underline{Bizonyítás:} }\hspace*{0.3cm}}}
\def\tetel{\large \textbf{Tétel: }}
\def\defi{\normalsize \textbf{Definíció: }}
\setlength\parindent{0pt}
\def\Z{\mathbb{Z}}
\def\Q{\mathbb{Q}}
\def\R{\mathbb{R}}
\def\N{\mathbb{N}}
\def\sume{\displaystyle\sum_{n=1}^{+\infty}}
\def\sumn{\displaystyle\sum_{n=0}^{+\infty}}
\def\biz{\normalsize{\underline{Bizonyítás:} }\hspace*{0.5cm}}
\def\tetel{\normalsize \textbf{\underline{Tétel}: }}
\def\narrow{\underset{n\rightarrow+\infty}{\longrightarrow}}
\def\limn{\displaystyle\lim_{n\to +\infty}}
\begin{center}
	{\LARGE\textbf{Analízis 2.}}\\[0.2cm]
	
	{\Large 1. Előadás jegyzet}\\[1cm]	
\end{center}
{\small A jegyzetet \textsc{Bauer Bence} készítette \textsc{Dr. Weisz Ferenc} előadása alapján.}\\[0.2cm]
Tantárgyi követelmények: \url{http://numanal.inf.elte.hu/~weisz/oktanyagok/Kov_An.pdf}\\[0.2cm]
\textbf{{\large Folytonos függvények}}\\[0.1cm]
\defi $f\in\R\to\R$ folytonos az $a\in D_f$ pontban, ha\\[0.2cm] $\forall\varepsilon>0,\exists\delta>0,\forall x\in D_f, |x-a|<\delta : |f(x)-f(a)|<\varepsilon$\\[0.1cm] Pl: \[\displaystyle f(x) = 
\left\{
\begin{gathered}
x: \quad x\in\Q\Rightarrow f\in C(0)\hspace{5.7cm} \\
\quad\quad-x: \quad x\notin\Q\Rightarrow f\notin C(0), a\neq 0\hspace{5.7cm}
\end{gathered}\right. \]\\[0.1cm]
\underline{Jelölés:}  $f\in C(a)$ folytonos az $a$ pontban.\\[0.2cm]
\tetel Folytonosság és határérték kapcsolata\\[0.1cm]
Ha $a\in D_f\cap D_f'$, akkor: \[ f\in C(a)\Leftrightarrow \exists\lim_a f \text{ és } \lim_a f=f(a)\]\\[0.1cm]
\biz Lásd az előző definíciót. $\quad\blacksquare$\\[0.3cm]
\defi $a\in D_f$ izolált pont, ha $\exists K(a)$, hogy $K(a)\cap D_f=\{a\}$\\[0.1cm] \textit{Állítás:} Ha $a\in D_f$, akkor $a\in D_f'$ vagy izolált pont.\\[0.1cm]\biz Triviális $\quad\blacksquare$\\[0.2cm] \tetel Ha $a\in D_f$ izolált, akkor $f\in C(a)$\\[0.1cm]\biz Triviális $\quad\blacksquare$\\[0.2cm] \tetel Hatványsor összegfüggvénye folytonos a konvergenciahalmaz belsejében.\\[0.1cm] \biz $\exists R\geq0:$ a hatványsor konvergens az ($a-R,a+R$) intervallumon.\\[0.1cm]Ezen kívül divergens, $x=a-R$, vagy $x=a+R$?\\[0.1cm]Hatványsor:\[f(x)=\sumn \alpha_n(x-a)^n\]
Valamint tanultuk korábban, hogy:\[\exists\lim_x f=f(x)\text{, ha }x\in(a-R,a+R)\quad\blacksquare\]\textbf{Következmény:} Az $exp,sin,cos,sh,ch$ függyvények folytonosak $\R$-en.\\[0.2cm]
\tetel Folytonosságra vonatkozó átviteli elv\\[0.1cm] Tfh. $a\in D_f$, ekkor:
\[f\in C(a)\Leftrightarrow\forall(x_n):\N\to D_f, \lim(x_n)=a:\quad\lim(f(x_n))=f(a)\]
\biz Ha $a\in D_f$, akkor előző + a tavalyi határértékre vonatkozó átviteli elv. Különben $a$ izolált pont. Ekkor mindkét oldal igaz.$\quad\blacksquare$\\[0.1cm] \tetel
\begin{itemize}
	\item Ha $f,g\in C(a)$, akkor $f+g\in C(a),\lambda f\in C(a), \lambda\in\R,f*g\in C(a)$. Ha még $g(a)\neq0$, akkor $\frac{f}{g}\in C(a)$.
	\item $g\in C(a), f\in C(g(a)), R_g\subset D_f$, akkor $f\circ g\in C(a)$.
\end{itemize}
\newpage\biz $(x_n):\N\to D_f, \lim(x_n)=a,\quad f,g\in C(a)\Rightarrow \lim(f(x_n))=f(a), \lim(g(x_n))=g(a)\\[0.1cm]\Rightarrow\lim(f(x_n)+g(x_n))= f(a)+g(a)\Rightarrow f+g\in C(a)\quad\blacksquare$\\[0.2cm]\defi $f$ folytonos $A$-n, ha $f\in C(a)\quad(\forall a\in A)$\\[0.3cm]
\textbf{{\large Korlátos és zárt intervallumon értelmezett függvények}}\\[0.2cm]
Ezután $f:[a,b]\to\R$\\[0.2cm] \defi Az $f:[a,b]\to\R$ függvénynek létezik abszolút maximuma (minimuma), ha\\[0.2cm] $\exists\alpha\in[a,b],\forall x\in [a,b]:f(x)\leq f(\alpha)\quad (f(x)\geq f(\alpha))$\\[0.1cm] Ahol $\alpha$ az abszolút maximum (minimum) hely.\\[0.2cm]
\tetel Ha $f:[a,b]\to\R \text{ folytonos, akkor } f $ korlátos. \\[0.1cm]	
\biz $f$ korlátos, ha $\exists K>0, \quad \forall x \in [a,b]: \quad |f(x)|\leq K$ \\[0.1cm]
Indirekt: Tegyük fel, hogy ez nem igaz, azaz\\[0.2cm]	
$\Rightarrow \forall K>0, \exists x \in [a,b]: |f(x)|> K \text{. Legyen a } K = n \text{. } \Rightarrow \forall n \in \N \text{, } \exists x_n \in [a,b]: |f(x_n)| > n \\[0.1cm] \Rightarrow x_n \in [a,b] \Rightarrow (x_n) \text{ korlátos} \Rightarrow $ Bolzano - Weierstrass tétel miatt \\[0.1cm] $  \exists (x_{n_k}) \text{ konvergens részsorozat } \Rightarrow \lim(x_{n_k}) =: \alpha \text{. Ekkor } \alpha \in [a,b] \\[0.1cm] \text{hiszen, ha } \alpha \notin [a,b] \text{, akkor } \exists \varepsilon > 0: [a,b] \cap K_\varepsilon (\alpha) = \emptyset \\[0.1cm] \Rightarrow \exists k_0 , \forall k \geq k_0: x_{n_k} \in K_\varepsilon (\alpha) $, viszont ez ellentmondás. \\[0.1cm] $ x_{n_k} \in [a,b] \Rightarrow \alpha \in [a,b] \Rightarrow f \in C(\alpha) $ \\[0.1cm] Alkalmazzuk az átviteli elvet, $ \lim(x_{n_k}) = \alpha \Rightarrow \lim f(x_{n_k}) = f(\alpha) \Rightarrow (f (x_{n_k})) $ konvergens. \\[0.1cm] $ \Rightarrow (f (x_{n_k}))$ korlátos. És így ellentmondásra juttotunk, hiszen: \\[0.1cm] $ |f (x_{n_k})| > n_k \Rightarrow (f (x_{n_k}))$ nem korlátos. $\blacksquare$\\[0.4cm]
\tetel (Weierstrass-tétel) Ha $ f:[a,b]\to\R\text{ folytonos, akkor } f$-nek létezik abszolút maximuma és minimuma is.\\[0.1cm]
\biz Ha $ f:[a,b]\to\R\text{ folytonos, akkor korlátos} \\[0.1cm] \Rightarrow M := $sup\{$f(x)\text{,ha } x \in [a,b]$\}, $m := $inf\{$f(x)\text{,ha } x \in [a,b]$\}, $M,m \in \R$\\[0.1cm] $\Rightarrow \forall n \geq 1$ -re, $\exists x \in [a,b]: M - \frac{1}{n} < f(x) \leq M\\[0.1cm] \Rightarrow \lim f(x_n) = M\Rightarrow (x_n)\text{ korlátos.}\\[0.1cm] \Rightarrow \exists (x_{n_k}) $ konvergens részsorozat$ \Rightarrow \lim x_{n_k} = \alpha, \alpha \in [a,b] \Rightarrow \text{átviteli elv, } f \in C(\alpha)\\[0.1cm] \Rightarrow \lim f(x_{n_k}) = f (\alpha). \\[0.1cm] \text{\textbf{De!} } \lim f(x_n) = M \Rightarrow \lim f (x_{n_k}) = M \Rightarrow M = f(\alpha) \\[0.1cm] m$ -re hasonló. $\blacksquare$
\end{document}