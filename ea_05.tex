\documentclass[a4paper,11pt]{article}
\usepackage[textwidth=170mm, textheight=230mm, inner=20mm, top=10mm, bottom=20mm]{geometry}
\usepackage[normalem]{ulem}
\usepackage[utf8]{inputenc}
\usepackage[T1]{fontenc}
\usepackage{physics}
\PassOptionsToPackage{defaults=hu-min}{magyar.ldf}
\usepackage[magyar]{babel}
\usepackage{amsmath, amsthm,amssymb,paralist,array, ellipsis, graphicx, float}

\begin{document}
\setlength\parindent{0pt}
\def\N{\mathbb{N}}
\def\Z{\mathbb{Z}}
\def\R{\mathbb{R}}
\def\biz{\normalsize{\underline{Bizonyítás:} }\hspace*{0.5cm}}
\def\tetel{\normalsize \textbf{\underline{Tétel}: }}
\def\defi{\normalsize \textbf{Definíció: }}
\def\limn{\displaystyle\lim_{n\to +\infty}}
\def\sumeu{\displaystyle\sum_{n=1}}
\def\sumun{\displaystyle\sum_{n=0}}
\def\sume{\displaystyle\sum_{n=1}^{\infty}}
\def\sumn{\displaystyle\sum_{n=0}^{\infty}}
\def\limh{\displaystyle\lim_{h\to0}}
\def\limxa{\displaystyle\lim_{x\to a}}
\def\limxatelj{\displaystyle\lim_{x\to a}\frac{f(x)-f(a)}{x-a}}
\def\pl{\textbf{Pl:}}
\def\rtr{\displaystyle\R\to\R}
\def\D{\displaystyle\mathcal{D}}
\def\lima{\displaystyle\lim_{a}}
\def\bizva{\quad\blacksquare}
\def\fda{f\in\D(a)}
\begin{center}
	{\LARGE\textbf{Analízis 2.}}\\[0.2cm]
	
	{\Large 5. Előadás jegyzet}\\[0.5cm]	
\end{center}
{\small A jegyzetet \textsc{Bauer Bence} készítette \textsc{Dr. Weisz Ferenc} előadása alapján.}\\[0.4cm]
\hspace*{0.5cm}\underline{Megj:}\hspace{1.7cm}$f(\xi)=\eta$\\[0.1cm]\hspace*{0.6cm} meredeksége: $f'(\xi) =m$\\[0.1cm]\hspace*{0.6cm} egyenlete: $y=mx+b\Rightarrow x=\frac{y-b}{m}\to$ meredeksége: $\frac{1}{m}$\\[0.1cm]\hspace*{0.6cm}$(f^{-1})' (\eta)=\frac{1}{m}=\frac{1}{f'(\xi)}$\\[0.2cm]
\textbf{\underline{Hatványsor deriváltja}}\\[0.2cm]
\tetel Legyen $\sumun\alpha_n(x-a)^n$ hatványsor konvergenciasugara $R>0$ és legyen \\$f(x):=\sumn\alpha_n(x-a)^n\quad x\in K_R(a)$. Ekkor $f\in\D(x_0)\quad\forall x_0\in K_R(a)$ és\\$f'(x_0)=\sume n*\alpha_n*(x_0-a)^{n-1},\quad$ahol $x_0\in K_R(a)$\\ \biz \underline{1. lépés:} Igazoljuk, hogy $\sumun n*\alpha_n*r^n$ abszolút konvergens $\forall0<r<R$\\Legyen $0<r<r'<R$ és $x=a+r'$\\[0.1cm]$x$-ben konvergens a hatványsor $\Rightarrow\sumun\alpha_n(r')^n$ konvergens $\Rightarrow\limn \alpha_n(r')^n=0\Rightarrow(\alpha_n(r')^n)$ korlátos\\[0.1cm] $\Rightarrow\exists M>0:|\alpha_n(r')^n|\leq M\Rightarrow |\alpha_n|\leq \frac{M}{(r')^n}\Rightarrow\sumun|n*\alpha_n*r^n|\leq M*\sumun n*$($\frac{r}{r'}$)$^n$\\ez konvergens, hiszen a gyökkritérium miatt\\ $\sqrt[n]{n*(\frac{r}{r'})^n}=\sqrt[n]{n}*$($\frac{r}{r'}$)$\to$($\frac{r}{r'}$)$<1 \Rightarrow\sumun n*\alpha_n*r^n$ abszolút konvergens\\$\Rightarrow\sumeu n*\alpha_n*r^{n-1}$ is abszolút konvergens $\Rightarrow\forall\varepsilon>0,\exists N:\displaystyle\sum_{n=N+1}^{\infty}|n*\alpha_n*r^{n-1}|<\frac{\varepsilon}{2}$ \\[0.1cm] \underline{2. lépés:}\\$\abs{\frac{f(x)-f(x_0)}{x-x_0}-\sume n*\alpha_n (x_0-a)^{n-1}}=\abs{\frac{\sumn\alpha_n (x-a)^n-\sumn\alpha_n (x_0-a)^n}{x-x_0}-\sume n*\alpha_n (x_0-a)^{n-1}}\leq\\[0.2cm]\leq\underbrace{\abs{ \displaystyle\sum_{n=1}^{N}\frac{\alpha_n(x-a)^n-\alpha_n(x_0-a)^n}{x-x_0}- \displaystyle\sum_{n=1}^{N}n*\alpha_n(x_0-a)^{n-1}}}_{(I)}+\underbrace{\abs{\displaystyle\sum_{n=N+1}^{\infty}\alpha_n\frac{(x-a)^n-(x_0-a)^n}{x-x_0}}}_{(II)}+\\[0.2cm]+\underbrace{\abs{\displaystyle\sum_{n=N+1}^{\infty}n*\alpha_n(x_0-a)^{n-1}}}_{(III)}= I+II+III$\\[0.1cm]Tfh. $\abs{x_0-a}<r<R$\\[0.1cm]Mivel $x\to x_0$ ezért feltehető, hogy $|x-a|<r\Rightarrow(III)\leq\displaystyle\sum_{n=1}^{N}n* \abs{\alpha_n}*r^{n-1}<\frac{\varepsilon}{2}\\[0.1cm](II)\leq\displaystyle\sum_{n=N+1}^{\infty}|\alpha_n|\abs{\frac{((x-a)-(x_0-a))((x-a)^{n-1}+(x-a)^{n-2}(x_0-a)+...+ (x_0-a)^{n-1})}{x-x_0}}=\\[0.3cm]=\displaystyle\sum_{n=N+1}^{\infty}|\alpha_n|\abs{(x-a)^{n-1}+(x-a)^{n-2}(x_0-a)+...+ (x_0-a)^{n-1}}=\displaystyle\sum_{n=N+1}^{\infty}|\alpha_n|*n*r^{n-1}<\frac{\varepsilon}{2}\\[0.3cm](I)\leq\sum_{n=1}^{N}|\alpha_n|\abs{\underbrace{\frac{(x-a)^n-(x_0-a)^n}{x-x_0}-n*(x_0-a)^{n-1}}_{\to0\text{, ha }x\to x_0}}\\[0.3cm]g(x)=(x-a)^n \Rightarrow$ a tört határértéke$\quad g'(x_0)=n*(x_0-a)^{n-1}\quad(x\to x_0) \\[0.2cm]\Rightarrow\exists\delta>0, (I)<\varepsilon$, ha $|x-x_0|<\delta\\[0.3cm] \Rightarrow\abs{\frac{f(x)-f(x_0)}{x-x_0}-\sume n*\alpha_n (x_0-a)^{n-1}}\leq \varepsilon+\varepsilon=2\varepsilon\quad(\text{ ha }|x-x_0|<\delta)\\[0.1cm] \Rightarrow lim\abs{\frac{f(x)-f(x_0)}{x-x_0}-\sume n*\alpha_n (x_0-a)^{n-1}}=0 \\[0.1cm]f\in\D(x_0)\text{ és }f'(x_0)=\sume n*\alpha_n*(x_0-a)^{n-1}\bizva$ \\[0.5cm]\textbf{\underline{Elemi függvények deriváltja}}
\begin{enumerate}
	\item $\exp\quad exp(x)=\displaystyle\sum_{n=0}^{\infty}\frac{x^n}{n!} \Rightarrow exp'(x)=\sum_{n=1}^{\infty}\frac{n*x^{n-1}}{n!}= \sum_{n=1}^{\infty}\frac{x^{n-1}}{(n-1)!}=exp(x)\quad(x\in\R)\\exp'=exp$
	\item $\sin x=\sumn(-1)^n\frac{x^{2n+1}}{(2n+1)!}\Rightarrow\sin'(x)=\sume (-1)^n\frac{(2n+1)*x^{2n}}{(2n+1)!}=\sumn(-1)^n\frac{x^{2n}}{(2n)!}=\cos x$
	\\[0.2cm] Hasonlóan $\cos'x=-\sin x,\quad ch'x=sh x,\quad sh'x=chx$
	\item $\ln=\exp^{-1},\quad\exp\xi=\eta\Rightarrow\\[0.1cm]\ln'(\eta)= \frac{1}{\exp'\xi}= \frac{1}{\exp\xi}=\frac{1}{\eta}\quad(\eta>0)$
	\item $a^x=exp_a(x),\quad a>0$\\[0.1cm]$exp_a'(x)=(exp(x*\ln a))'= exp(x*\ln a)*\ln a=a^x*\ln a$
	\item $\log_a=(exp_a)^{-1},\quad exp_a(\xi)=\eta\quad(a>0,a\neq1) \\[0.3cm]\log_a'(\eta)=\frac{1}{exp_a'(\xi)}=\frac{1}{a^x*\ln a}= \frac{1}{\eta*\ln a}\quad(\eta>0,a>0,a\neq1)$
	\item $f(x)=x^\alpha\quad(\alpha\in\R,x>0)\\[0.3cm](x^\alpha)'= (exp(\alpha*\ln x))'=exp(\alpha*\ln x)*\alpha*\frac{1}{x}= \alpha*x^\alpha*\frac{1}{x}=\alpha*x^{\alpha-1}$
	\item $F(x)=f(x)^{g(x)}=exp(\ln(f(x)^{g(x)}))=exp(g(x)*\ln*f(x))=
	\\[0.2cm]=exp(g(x)*\ln*f(x))*(g'(x)*\ln f(x)+g(x)*\frac{1}{f(x)}*f'(x))$
\end{enumerate}
\textbf{\underline{Egyoldali derivált}}\\[0.2cm]
$f(x)=\abs{x}\quad f\notin\D(0)\\[0.2cm]\displaystyle\lim_{x\to0+0} \frac{f(x)-f(0)}{x-0}=\lim_{x\to0+0}\frac{x}{x}=1\\[0.2cm]\lim_{x\to0-0} \frac{f(x)-f(0)}{x}=\lim_{x\to0-0}\frac{-x}{x}=-1$\\[0.2cm]
\defi $f\in\rtr,\quad a\in\D_f\text{  és  }\exists\delta>0:[a,a+\delta) \subset\D_f$\\[0.2cm]Ekkor: $f$ jobbról deriválható $a$-ban, ha\\[0.2cm] $f_+'(a)=\displaystyle\lim_{x\to a+0}\frac{f(x)-f(a)}{x-a}$ határérték létezik és véges.\\[0.2cm]Hasonló az $f_-'(a)$ definíciója.\newpage
\tetel $f\in\rtr,\quad a\in int\D_f\quad$Ekkor:\\[0.2cm] $\fda\Leftrightarrow f_+'(a)\text{  és  }f_-'(a)$ léteznek és egyenlőek.\\[0.3cm]
\textbf{\underline{Többször deriválható függyvények}}\\[0.2cm]
\defi $f$ kétszer deriválható $a$-ban, ha $\exists K(a)$, hogy $f\in\D(K(a))$ és $f'\in\D(a)$\\[0.2cm] Jelölés: $f''(a)=(f')'(a),\quad(f\in\D^2(a))$ \\[0.2cm]\defi Az $f$ függvény $(n+1)$-szer differenciálható $a$-ban, ha 
\\[0.2cm]$\exists K(a)$, hogy $f\in\D^n(K(a))$ és $f^{(n)}\in\D(a)$\\[0.2cm] Jelölés: $f^{(n+1)}(a)=(f^{(n)})'(a),\quad(f\in\D^{n+1})(a)\\[0.1cm]f^{(0)}=f$
\\[0.2cm]\defi $f$ végtelenszer deriválható $a$-ban, ha $\forall n\in\N :f\in\D^n(a)$\\[0.2cm]\tetel (Leibniz)\\[0.2cm] $f,g\in\D^n(a)\Rightarrow f*g\in\D^n(a)$ és $(f*g)^{(n)}(a)=\displaystyle\sum_{k=0}^{n}\binom{n}{k}*f^{(k)} (a)*g^{(n-k)}(a)$\\[0.2cm]\biz Teljes indukcióval\\[0.2cm]$\underline{n=0:}\quad (f*g) (a)=f(a)*g(a)$\\[0.2cm]$\underline{n=1:}\quad(f*g)'(a)=1*f(a)*g'(a)+1*f'(a)* g(a)$\\[0.2cm]Folytatás Hf.\\[0.4cm]\tetel Tfh. $f(x)=\displaystyle \sum_{k=0}^{\infty}\alpha_k(x-a)^k,\quad x\in K_R(a)\\[0.2cm]\text{Ekkor: } f\in\D^n(x_0) \quad n\in\N\text{  és}\\[0.2cm]f^{(n)}(x_0)=\sum_{k=n}^{\infty} \alpha_k*k*(k-1)*...*(k-n+1)*(x_0-a)^{k-n}\quad(x_0\in K_R(a))$\\[0.4cm]Továbbá: $f^{(n)}(a)=\alpha_n*n!$\\[0.2cm]\biz Hf. Teljes indukcióval.
\end{document}