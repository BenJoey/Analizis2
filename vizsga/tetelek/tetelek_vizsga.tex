\documentclass[a4paper,11pt]{article}
\usepackage[textwidth=170mm, textheight=230mm, inner=20mm, top=10mm, bottom=20mm]{geometry}
\usepackage[normalem]{ulem}
\usepackage[utf8]{inputenc}
\usepackage[T1]{fontenc}
\usepackage{physics}
\PassOptionsToPackage{defaults=hu-min}{magyar.ldf}
\usepackage[magyar]{babel}
\usepackage{amsmath, amsthm,amssymb,paralist,array, ellipsis, graphicx, float}

\begin{document}
	\def\R{\mathbb{R}}
	\def\N{\mathbb{N}}
	\def\sume{\displaystyle\sum_{n=1}^{\infty}}
	\def\sumn{\displaystyle\sum_{n=0}^{\infty}}
	\def\sumeu{\displaystyle\sum_{n=1}}
	\def\sumun{\displaystyle\sum_{n=0}}
	\def\cab{\in C[a,b]}
	\def\dab{\in\D(a,b)}
	\def\biz{\normalsize{\textbf{\underline{Bizonyítás:} }\hspace*{0.5cm}}}
	\def\tetel{\normalsize \textbf{Tétel: }}
	\def\limn{\displaystyle\lim_{n\to +\infty}}
	\def\limh{\displaystyle\lim_{h\to0}}
	\def\limxa{\displaystyle\lim_{x\to a}}
	\def\limxatelj{\displaystyle\lim_{x\to a}\frac{f(x)-f(a)}{x-a}}
	\def\rtr{\displaystyle\R\to\R}
	\def\D{\displaystyle\mathcal{D}}
	\def\lima{\displaystyle\lim_{a}}
	\def\bizva{\quad\blacksquare}
	\def\fda{f\in\D(a)}
	\def\limaj{\displaystyle\lim_{a+0}}
	\def\limnj{\displaystyle\lim_{0+0}}
	\def\Rv{\overline{\mathbb{R}}}
	\def\fabr{f:(a,b)\to\R}
	\def\itr{I\to\R}
	\def\prfv{primitív függvény}
	\def\rab{R[a,b]}
	\def\te{\tau_1}
	\def\tk{\tau_2}
	\def\ftau{(f,\tau)}
	\def\kab{K[a,b]}
	\def\fab{F[a,b]}
	\def\sumi{\sum\limits_{i=1}^{n}}
	\def\intv{[x_{i-1},x_i]}
	\def\intab{\int\limits_{a}^{b}}	
\begin{center}
	{\LARGE \textbf{Analízis 2}}\\[0.2cm]
	{\large Tételbizonyítások vizsgára}
\end{center}
{\small A jegyzetet \textsc{Bauer Bence} készítette \textsc{Dr. Weisz Ferenc} előadása alapján.}
\section{Korlátos zárt intervallumon értelmezett függvény korlátos is.}
\tetel Ha $f:[a,b]\to\R \text{ folytonos, akkor } f $ korlátos. \\[0.1cm]	
\biz $f$ korlátos, ha $\exists K>0, \quad \forall x \in [a,b]: \quad |f(x)|\leq K$ \\[0.1cm]
Indirekt: Tegyük fel, hogy ez nem igaz, azaz\\[0.2cm]	
$\Rightarrow \forall K>0,\exists x \in[a,b]:|f(x)|> K\text{. Legyen a }K=n\text{. } 
\Rightarrow\forall n\in\N\text{, }\exists x_n\in[a,b]:|f(x_n)|>n\\[0.1cm] 
\Rightarrow x_n\in[a,b]\Rightarrow(x_n)\text{ korlátos}\Rightarrow$ Bolzano -
Weierstrass tétel miatt \\[0.1cm] $\exists(x_{n_k})\text{ konvergens részsorozat }
\Rightarrow\lim(x_{n_k})=:\alpha\text{. Ekkor }\alpha\in[a,b]\\[0.1cm]
\text{hiszen, ha }\alpha\notin[a,b]\text{, akkor }\exists\varepsilon>0:[a,b]\cap
K_\varepsilon(\alpha)=\emptyset\\[0.1cm]\Rightarrow\exists k_0,\forall k\geq k_0:
x_{n_k}\in K_\varepsilon(\alpha)$, viszont ez ellentmondás. \\[0.1cm]$x_{n_k}\in
[a,b]\Rightarrow\alpha\in[a,b]\Rightarrow f\in C(\alpha)$\\[0.1cm] Alkalmazzuk az
átviteli elvet, $\lim(x_{n_k})=\alpha\Rightarrow\lim f(x_{n_k})=f(\alpha)
\Rightarrow(f (x_{n_k}))$ konvergens. \\[0.1cm] $\Rightarrow(f (x_{n_k}))$
korlátos. És így ellentmondásra juttotunk, hiszen: \\[0.1cm] $|f(x_{n_k})|>n_k
\Rightarrow(f (x_{n_k}))$ nem korlátos.$\bizva$
\section{Weierstrass-tétel}
\tetel Ha $ f:[a,b]\to\R\text{ folytonos, akkor } f$-nek
létezik abszolút maximuma és minimuma is.\\[0.1cm]
\biz Ha $ f:[a,b]\to\R\text{ folytonos, akkor korlátos} \\[0.1cm]\Rightarrow 
M:=$sup\{$f(x)\text{,ha } x\in[a,b]$\}, $m:=$inf\{$f(x)\text{,ha }x\in[a,b]$\},
$M,m\in\R$\\[0.1cm] $\Rightarrow\forall n\geq1$-re, $\exists x\in[a,b]:
M -\frac{1}{n}<f(x)\leq M\\[0.1cm]\Rightarrow\lim f(x_n)=M\Rightarrow(x_n)
\text{ korlátos.}\\[0.1cm]\Rightarrow\exists(x_{n_k})$ konvergens részsorozat$ 
\Rightarrow\lim x_{n_k}=\alpha,\alpha\in[a,b]\Rightarrow\text{átviteli elv, }
f\in C(\alpha)\\[0.1cm]\Rightarrow\lim f(x_{n_k})=f(\alpha).\\[0.1cm]
\text{\textbf{De!} }\lim f(x_n)=M\Rightarrow\lim f(x_{n_k})=M\Rightarrow
M=f(\alpha)\\[0.1cm]m$ -re hasonló.$\bizva$
\section{Bolzano-tétel}
\tetel Tfh. $f: [a,b] \to \R$ folytonos. Ha $f(a)\cdot f(b) < 0$ akkor
$\exists x \in [a,b]: f(x) = 0$\\[0.1cm]
\biz Legyen $ [x_0,y_0] = [a,b] $ és tfh. $f(a) < 0$ és $f(b) > 0 $\\[0.1cm] Legyen
$z_0 := \frac{x_0+y_0}{2}$, ekkor 3 eset lehetséges: 
\begin{enumerate}
	\item $f(z_0) = 0$ $\checkmark$
	\item $f(z_0) < 0$, ekkor legyen $[x_1,y_1] := [z_0,y_0]$
	\item $f(z_0) > 0$, ekkor legyen $[x_1,y_1] := [x_0,z_0]$
\end{enumerate}
Ezt az eljárást folytatva véges sok lépésben kapunk $\xi$-t amelyre $f(\xi) = 0$,
ha nem akkor kapunk egy ($[x_n,y_n]$) intervallum sorozatot, amelyre a következők
igazak:
\begin{enumerate}
	\item $[x_{n+1},y_{n+1}]\subset[x_n,y_n]$
	\item $f(x_n) < 0,\quad f(y_n) > 0$
	\item $y_n - x_n = \frac{y_0-x_0}{2^n}$
\end{enumerate}
\newpage
A Cantor-tétel miatt \[ \exists\xi\in\bigcap_{n=0}^{+\infty}[x_n,y_n]\text{ Mivel }
y_n-x_n=\frac{y_0-x_0}{2^n}\to0\hspace*{0.3cm}(n\to\infty)\text{ ,ezért  }\exists!
\xi\in\bigcap_{n=0}^{+\infty}[x_n,y_n] \]
Továbbá $0\leq\xi-x_n\leq y_n-x_n\to0\Rightarrow\underline{\lim(x_n)=\xi}$ ,és
$\quad y_n-\xi\leq y_n-x_n\to0\Rightarrow\underline{\lim(y_n)=\xi}$\\[0.2cm]
Tudjuk, hogy $f(x_n)<0$ és $\lim(x_n)=\xi$ és $f\in C(\xi)$ ,ezért az átviteli elv
miatt\\ $\lim f(x_n)=f(\xi)\Rightarrow f(\xi)\leq0$\\[0.1cm] Hasonlóan $f(y_n)>0,
\quad\lim(y_n)=\xi\Rightarrow\lim f(y_n)=f(\xi)$\\itt: $f(y_n)>0$ ezért $f(\xi)
\geq0\Rightarrow f(\xi)=0\bizva$
\section{Heine-tétel}
\tetel Ha $f:[a,b]\to\R$ folytonos, akkor $f$ egyenletesen folytonos.\\[0.1cm]
\biz (Indirekt) Tfh. $f$ nem egyenletesen folytonos.\\[0.1cm]
$\Rightarrow\exists\varepsilon>0,\forall\delta>0,\exists x,y\in[a,b]:|x-y|<\delta:
|f(x)-f(y)|\geq\varepsilon$\\[0.1cm] Legyen $\delta=\frac{1}{n}\quad n\in\N_+
\Rightarrow\exists\varepsilon>0,\forall n\in\N_+:\exists x_n,y_n\in[a,b]:|x_n-y_n|
<\frac{1}{n}:|f(x_n)-f(y_n)|\geq\varepsilon$ \\[0.1cm] Tekintsük az $(x_n): 
\N\to[a,b]\text{ sorozatot}\Rightarrow (x_n)$ korlátos.\\[0.1cm]
Bolzano-Weierstrass kiv. tétel miatt $\exists(x_{n_k})$ konvergens részsorozat,
azaz:\\[0.1cm] $\lim(x_{n_k})=:\alpha,\quad\alpha\in[a,b]$\\[0.1cm]\textbf{De!}
$|y_{n_k}-\alpha|\leq|y_{n_k}-x_{n_k}|+|x_{n_k}-\alpha|<\frac{1}{n_k}
+|x_{n_k}-\alpha|\to0\quad$ azaz $\lim(y_{n_k})=\alpha$\\[0.1cm] $f\in C(\alpha)$
átviteli elv miatt\\[0.1cm]$\lim(f(x_{n_k}))=f(\alpha)$ és $\lim(f(y_{n_k}))=
f(\alpha)\Rightarrow\lim(f(x_{n_k})-f(y_{n_k}))=0$\\[0.1cm] viszont ez
ellentmondás, azzal, hogy $|f(x_{n_k})-f(y_{n_k})|\geq\varepsilon\bizva$
\section{Az inverzfüggvény folytonossága}
\tetel Ha $f:[a,b]\to\R$ folytonos és injektív, akkor $f^{-1}$ is folytonos.\\[0.1cm]
\biz \underline{1. lépés:} $f^{-1}$  $\exists$ Indirekt.\\[0.1cm] Tfh. $f^{-1}$ nem
folytonos. $\Rightarrow\exists y_{0}\in R_f, f^{-1}\notin C(y_0)\Rightarrow$
átviteli elv.\\[0.1cm] $\exists y_n\in R_f,\lim(y_n)=y_0$ : $\lim(f^{-1}(y_n))\neq
f^{-1}(y_0)$\\[0.1cm] Legyen $x_n=f^{-1}(y_n),n\in\N\Rightarrow\lim(x_n)\neq
x_0\Rightarrow\exists\delta>0$: $\{n:|x_n-x_0|\geq\delta\}$ végtelen.\\[0.1cm]
Legyen $n_k$ indexsorozat, hogy: $|x_{n_k}-x_0|\geq\delta\\[0.1cm](x_{n_k}):
\N\to[a,b]\Rightarrow(x_{n_k})$ korlátos. $\Rightarrow\exists$ konvergens
részsorozat.\\[0.1cm] $(x_{n_k})':\quad\lim(x_{n_k})':=\alpha\quad$\textbf{De!}
$\quad|(x_{n_k})'-x_0|\geq\delta\Rightarrow|\alpha-x_0|\geq\delta\Rightarrow\alpha
\neq x_0$\\[0.1cm]\underline{2. lépés:} $f\in C(\alpha)\quad\alpha\in[a,b]$\\[0.1cm]
átviteli elv $\Rightarrow\lim\underbrace{f(x_{n_k})'}_{(y_{n_k})'}=
f(\alpha)\Rightarrow\lim(y_{n_k})'=f(\alpha)$\\[0.1cm]\textbf{De!}
$\lim(y_{n_k})'=y_0=f(x_0)\Rightarrow f(\alpha)=f(x_0)\quad f$ injektív.
$\Rightarrow\alpha=x_0\quad$ Ez ellentmondás. $\bizva$
\newpage
\section{Folytonos invertálható függvény jellemzése a monotonitással.}
\tetel $f:[a,b]\to\R$ folytonos és injektív $\Rightarrow f$ szig. mon.\\[0.1cm]
\biz Ha $f(a)<f(b)$, ekkor $f$ szig. mon. nő\\[0.1cm] \textbf{1.} Igazoljuk, hogy
$f(a)=min\{f(x):x\in[a,b]\}$ és $f(b)=max\{f(x):x\in[a,b]\}$\\[0.1cm] Csak az
elsőt. Indirekten, Tfh:\\[0.1cm] $f(a)>min f$ (< nem lehet) Weierstrass-tétel
$\Rightarrow\exists\alpha\in[a,b]:f(\alpha)=min f\quad\alpha\neq a,b$\\[0.1cm]
Tekintsük az $f:[\alpha,b]\to\R$ függvényt. A Bolzano-tétel miatt $c=f(a)
\in(f(\alpha),f(b))$-hoz is\\[0.1cm] $\exists\xi\in[\alpha,b]:f(a)=f(\xi)\quad f$
injektív $\Rightarrow a=\xi\quad$ Ellentmondás.\\[0.1cm]\textbf{2.} Igazoljuk, ha
$x_1<x_2\Rightarrow f(x_1)<f(x_2)\quad(x_1,x_2\in[a,b])$\\[0.1cm]Indirekt, Tfh:
$f(x_1)>f(x_2)$\\[0.1cm]Ekkor $f(x_1)\in(f(x_2),f(b))\quad$ Tekintsük az $f:[x_2,b]
\to\R$ függvényt.\\[0.1cm]$c=f(x_1)$-hez is $\exists\xi\in(x_2,b):
f(x_1)=f(\xi)\quad f$ injektív $\Rightarrow x_1=\xi\quad$ Ellentmondás.$\bizva$
\section{Differenciálható függvények összege, szorzata, hányadosa.}
\tetel Legyen $f,g\in\rtr,a\in int(\D_f\cap\D_g),\quad f,g\in\D(a)$ Ekkor:
\\[0.2cm]\hspace*{0.7cm}\textbf{i,} $f+g\in\D(a)\text{ és }(f+g)'(a)=f'(a)+g'(a)$
\\[0.1cm]\hspace*{0.7cm}\textbf{ii,} $f\cdot g\in\D(a)\text{ és }(f\cdot g)'(a)=
f'(a)\cdot g(a)+f(a)\cdot g'(a)$\\[0.1cm]\hspace*{0.7cm}\textbf{iii,} Ha $g(a)\neq0$, akkor 
$\frac{f}{g}\in\D(a)\text{ és }(\frac{f}{g})'(a)=\frac{(f'(a)\cdot g(a)-f(a)\cdot g'(a))}{g^2(a)}$
\\[0.1cm]\biz\\[0.1cm]\hspace*{0.3cm}\textbf{i,} $a\in int(\D_f\cap\D_g)=int\D_{f+g}
\\[0.2cm]\limxa \frac{(f+g)(x)-(f+g)(a)}{x-a}=\limxatelj+\limxa
\frac{g(x)-g(a)}{x-a}=f'(a)+g'(a) \\[0.2cm]\Rightarrow(f+g)'(a)=f'(a)+g'(a)$
\\[0.3cm]\hspace*{0.3cm}\textbf{ii,} $\limxa\frac{(f\cdot g)(x)-(f\cdot g)(a)}{x-a}=
\limxa\frac{f(x)\cdot g(x)-f(a)\cdot g(a)}{x-a}=\\[0.4cm]=\limxa\frac{f(x)\cdot
g(x)-g(x)\cdot f(a)+g(x)\cdot f(a)-f(a)\cdot g(a)}{x-a}=\\[0.4cm]=\limxa
g(x)\cdot\underbrace{\frac{f(x)-f(a)}{x-a}}_{\to f'(a)}+\limxa
f(a)\cdot\underbrace{\frac{g(x)-g(a)}{x-a}}_{\to g'(a)}\longrightarrow f'(a)\cdot
g(a)+f(a)\cdot g'(a)\quad(x\to a)\\[0.2cm]\Rightarrow(f\cdot g)'(a)= f'(a)\cdot
g(a)+f(a)\cdot g'(a)$\\[0.3cm]\hspace*{0.3cm}\textbf{iii,} Először igazoljuk, hogy
$(\frac{1}{g})'(a)=-\frac{g'(a)}{g^2(a)}\\[0.2cm]\limxa\frac{\frac{1}{g}(x)-
\frac{1}{g}(a)}{x-a}=\limxa\frac{\frac{1}{g(x)}-\frac{1}{g(a)}}{x-a}=\limxa
\frac{g(a)-g(x)}{g(x)\cdot g(a)\cdot(x-a)}=\\[0.4cm]=\limxa
\underbrace{(-\frac{1}{g(x)\cdot g(a)})}_{\to\frac{-1}{g^2(a)}}
\cdot\underbrace{(\frac{g(x)-g(a)}{x-a})}_{\to g'(a)}\longrightarrow
-\frac{g'(a)}{g^2(a)}\quad(x\to a)\\[0.4cm]\Rightarrow 
(\frac{f}{g})'(a)=(f\cdot\frac{1}{g})'(a)=f'(a)\cdot\frac{1}{g(a)}+f(a)\cdot
(\frac{1}{g})'(a)= \frac{f'(a)}{g(a)}-\frac{f(a)\cdot g'(a)}{g^2(a)}=\frac{f'(a)\cdot g(a)-f(a)\cdot g'(a)}{g^2(a)} \\[0.25cm]\bizva$
\newpage
\section{Differenciálható függvények kompozíciója.}
\tetel $f,g\in\rtr,R_g\subset D_f,g \in\D(a),f\in\D(g(a))$, ekkor\\[0.1cm] $f\circ
g\in\D(a)\text{ és }(f\circ g)'(a)= f'(g(a))\cdot g'(a)\\[0.2cm]$\biz$g\in\D(a)
\Rightarrow a\in int\D_g\Rightarrow int\D_{f\circ g}\\[0.1cm]
g\in\D(a)\Rightarrow\exists\varepsilon_1\in\rtr,\lima \varepsilon_1=0
\text{ és }g(x)-g(a)=g'(a)(x-a)+\varepsilon_1(x)(x-a)\quad(x\in D_f)\\[0.2cm]
f\in\D(g(a))\Rightarrow\exists\varepsilon_2\in\rtr,\displaystyle\lim_{g(a)}
\varepsilon_2=0\text{ és }f(y)-f(g(a))=f'(g(a))\cdot(y-g(a))+\varepsilon_2(y)
\cdot(y-g(a))$\\[0.1cm]Legyen $y=g(x)$\\[0.1cm]$f(g(x))-f(g(a))=f'(g(a))\cdot 
(g(x)-g(a))+ \varepsilon_2(g(x))\cdot (g(x)-g(a))=\\[0.2cm]=f'(g(a))\cdot 
(g'(a)(x-a)+\varepsilon_1(x)(x-a))+ \varepsilon_2(g(x))\cdot(g'(a)(x-a)
+\varepsilon_1(x)(x-a))=\\[0.2cm]=f'(g(a))\cdot g'(a)\cdot (x-a)+ (x-a)\cdot 
\underbrace{(f'(g(a))\cdot\varepsilon_1(x)+\varepsilon_2(g(x))\cdot 
g'(a)+\varepsilon_1(x)\cdot  \varepsilon_2(g(x)))}_{\varepsilon(x)}$
\\[0.2cm]$\varepsilon_1\to0,\quad(x\to a)\\[0.1cm] g(x)\to g(a)\\[0.1cm]
\Rightarrow\limxa\varepsilon_2(g(x))=\displaystyle\lim_{g(a)}\varepsilon_2=0
\Rightarrow\lima\varepsilon=0\\[0.2cm]\Rightarrow(f\circ g)'(a)= f'(g(a))\cdot 
g'(a)\bizva$
\section{Az inverz függvény deriváltja.}
\tetel Legyen $f:(a,b)\to\R$, szig. mon. növő és folytonos függvény.\\[0.1cm]Ha 
$\xi\in(a,b),f\in\D(\xi) \text{ és }f'(\xi)\neq0$, akkor\\[0.1cm] 
$(f^{-1})\in\D(\eta)\text{ és }(f^{-1})'(\eta)=\frac{1}{f'(\xi)}
\text{, ahol }\eta=f(\xi)$\\[0.2cm]\biz $f:(a,b)\to\R\text{ folytonos }
\Rightarrow R_f$ intervallum.\\[0.2cm] $f$ szig. mon. növő $\Rightarrow R_f$ 
nyílt intervallum $\Rightarrow\eta\in int R_f\quad\quad f^{-1}:R_f\to D_f$
\\[0.2cm] $\displaystyle\lim_{y\to\eta}\frac{f^{-1}(y)-f^{-1}(\eta)}{y-\eta}=
\displaystyle\lim_{x\to\xi}\frac{x-\xi}{f(x)-f(\xi)}=\displaystyle\lim_{x\to\xi}
\frac{1}{\frac{f(x)-f(\xi)}{x-\xi}}\longrightarrow\frac{1}{f'(\xi)}\quad
(x\to\xi)$\\[0.2cm] Legyen $f(x)=y\Leftrightarrow x=f^{-1}(y)\quad
\quad\xi=f^{-1}(\eta)$\\[0.2cm] Ui. $x\to\xi$, mert $y\to\eta:\quad 
f:(a,b)\to\R$ folytonos és injektív $\Rightarrow f^{-1}$ folytonos $\Rightarrow 
f^{-1}(y)\to f^{-1}(\eta)\\[0.1cm]\Rightarrow x\to\xi\\[0.2cm](f^{-1})'(\eta)
=\displaystyle\lim_{y\to\eta}\frac{f^{-1}(y)-f^{-1}(\eta)}{y-\eta}=\lim
\limits_{x\to\xi}\frac{1}{\frac{f(x)-f(\xi)}{x-\xi}}= \frac{1}{f'(\xi)}\bizva$
\section{Hatványsor összegfüggvényének deriváltja}
\tetel Legyen $\sumun\alpha_n(x-a)^n$ hatványsor konvergenciasugara $R>0$ és legyen
\\$f(x):=\sumn\alpha_n(x-a)^n\quad x\in K_R(a)$. Ekkor $f\in\D(x_0)\quad\forall
x_0\in K_R(a)$ és\\$f'(x_0)=\sume n\cdot\alpha_n\cdot(x_0-a)^{n-1},\quad$ahol $x_0\in K_R(a)$\newpage
\biz \underline{1. lépés:} Igazoljuk, 
hogy $\sumun n\cdot\alpha_n\cdot r^n$ abszolút konvergens $\forall0<r<R$\\Legyen 
$0<r<r'<R$ és $x=a+r'$\\[0.1cm]$x$-ben konvergens a hatványsor $\Rightarrow
\sumun\alpha_n(r')^n$ konvergens $\Rightarrow\limn\alpha_n(r')^n=0
\Rightarrow(\alpha_n(r')^n)$ korlátos\\[0.1cm] $\Rightarrow\exists M>0:
|\alpha_n(r')^n|\leq M\Rightarrow |\alpha_n|\leq \frac{M}{(r')^n}
\Rightarrow\sumun|n\cdot\alpha_n\cdot r^n|\leq M\cdot\sumun n\cdot
$($\frac{r}{r'}$)$^n$\\ez konvergens, hiszen a gyökkritérium miatt\\
$\sqrt[n]{n\cdot(\frac{r}{r'})^n}=\sqrt[n]{n}\cdot$($\frac{r}{r'}$)$\to$
($\frac{r}{r'}$)$<1 \Rightarrow\sumun n\cdot\alpha_n\cdot r^n$ abszolút 
konvergens\\$\Rightarrow\sumeu n\cdot\alpha_n\cdot r^{n-1}$ is abszolút 
konvergens $\Rightarrow\forall\varepsilon>0,\exists N:\displaystyle
\sum_{n=N+1}^{\infty}|n\cdot\alpha_n\cdot r^{n-1}|< 
\frac{\varepsilon}{2}$\\[0.1cm]\underline{2. lépés:}\\
$\abs{ \frac{f(x)-f(x_0)}{x-x_0}-\sume n\cdot \alpha_n (x_0-a)^{n-1}}
=\abs{\frac{\sumn\alpha_n (x-a)^n-\sumn\alpha_n (x_0-a)^n}{x-x_0}-\sume 
n\cdot\alpha_n (x_0-a)^{n-1}}\leq\\[0.2cm]\leq\underbrace{\abs{ 
\displaystyle\sum_{n=1}^{N}\frac{\alpha_n(x-a)^n-\alpha_n(x_0-a)^n}{x-x_0}- 
\displaystyle\sum_{n=1}^{N}n\cdot\alpha_n(x_0-a)^{n-1}}}_{(I)}+\underbrace
{\abs{\displaystyle\sum_{n=N+1}^{\infty}\alpha_n\frac{(x-a)^n-(x_0-a)^n}{x-x_0}}}
_{(II)}+\\[0.2cm]+\underbrace{\abs{\displaystyle\sum_{n=N+1}^{\infty}n\cdot\alpha
_n(x_0-a)^{n-1}}}_{(III)}= (I)+(II)+(III)$\\[0.1cm]Tfh. $\abs{x_0-a}<r<R$
\\[0.1cm]Mivel $x\to x_0$ ezért feltehető, hogy $|x-a|<r\Rightarrow(III)
\leq\displaystyle \sum_{n=N+1}^{\infty}n\cdot \abs{\alpha_n}\cdot r^{n-1}<
\frac{\varepsilon}{2} \\[0.1cm](II)\leq\displaystyle
\sum_{n=N+1}^{\infty}|\alpha_n|\abs{\frac{((x-a)-(x_0-a))((x-a)^{n-1}+(x-a)^{n-2}
(x_0-a)+...+ (x_0-a)^{n-1})}{x-x_0}}=\\[0.3cm]=\displaystyle
\sum_{n=N+1}^{\infty}|\alpha_n|\abs{(x-a)^{n-1}+(x-a)^{n-2}(x_0-a)+...+ 
(x_0-a)^{n-1}}=\displaystyle \sum_{n=N+1}^{\infty}|\alpha_n|\cdot n\cdot 
r^{n-1}<\frac{\varepsilon}{2} \\[0.3cm](I)\leq\sum_{n=1}^{N}|\alpha_n|
\abs{\underbrace{\frac{(x-a)^n-(x_0-a)^n}{x-x_0}-n\cdot(x_0-a)^{n-1}}_
{\to0\text{, ha }x\to x_0}}\\[0.3cm]g(x)=(x-a)^n \Rightarrow$ a tört határértéke
$\quad g'(x_0)=n\cdot(x_0-a)^{n-1}\quad(x\to x_0) \\[0.2cm]\Rightarrow\exists
\delta>0, (I)<\varepsilon$, ha $|x-x_0|<\delta\\[0.3cm] \Rightarrow
\abs{\frac{f(x)-f(x_0)}{x-x_0}-\sume n\cdot\alpha_n (x_0-a)^{n-1}}< 
\varepsilon+\varepsilon=2\varepsilon\quad(\text{ ha }|x-x_0|<\delta)\\[0.1cm] 
\Rightarrow \lim\abs{\frac{f(x)-f(x_0)}{x-x_0}-\sume n\cdot\alpha_n 
(x_0-a)^{n-1}}=0 \\[0.1cm]f\in\D(x_0)\text{ és }f'(x_0)=\sume n\cdot\alpha_n 
\cdot(x_0-a)^{n-1}\bizva$
\section{A differenciálszámítás középértéktételei \\ (Rolle-, Cauchy-, Lagrange-tétel).}
\textbf{1. Rolle-tétel}\\[0.1cm]\tetel Tfh. $f\cab$ és $f\dab$\\[0.2cm]Ha
$f(a)=f(b)$, ekkor $\exists\xi\in(a,b):f'(\xi)=0$\\[0.2cm]\biz $f\cab\Rightarrow$
Weierstrass-tétel miatt\\[0.2cm]$\Rightarrow\exists\alpha\in[a,b]:f(\alpha)=
\min\limits_{[a,b]}f=:m$ és $\exists\beta\in[a,b]:f(\beta)= \max\limits_{[a,b]}f=:M$
\\[0.2cm]\underline{1. lépés:} Tfh. $m=M\Rightarrow f=m\quad([a,b]$-n$)$, a függvény
konstans $\Rightarrow f'=0\quad[a,b]$-n\\[0.2cm]\underline{2. lépés:} $m\neq M$ és
$m\neq f(a)=f(b)\Rightarrow m=f(\alpha)\Rightarrow\alpha\neq a,b
\Rightarrow\alpha\in(a,b)\\[0.2cm]\Rightarrow\alpha$-ban lokális minimum van.
$\Rightarrow f'(\alpha)=0$\\[0.2cm]\underline{3. lépés:} Tfh. $m\neq M$ és $m=f(a)
=f(b)$\\[0.2cm]$\Rightarrow M\neq f(a)=f(b)\Rightarrow\beta\neq a,b\Rightarrow\beta
\in(a,b)\Rightarrow\beta$-ban lokális maximum van $\Rightarrow f'(\beta)=0\bizva$
\\[0.2cm]\textbf{2. Cauchy-tétel}\\[0.1cm]\tetel Tfh. $f,g\cab$, $f,g\dab$ és
$g'(x)\neq0\quad(x\in(a,b))$\\[0.2cm]Ekkor: $\exists\xi\in(a,b):
\frac{f(b)-f(a)}{g(b)-g(a)}=\frac{f'(\xi)}{g'(\xi)}$\\[0.2cm]\biz
$g(b)\neq g(a)$, hiszen különben $\exists\xi\in(a,b):g'(\xi)=0$\\[0.2cm]Válasszuk meg
$\lambda$-t úgy, hogy az $F:=f-\lambda g$ függvényre alkalmazhassuk a
Rolle-tételt\\[0.2cm]$F\cab,F\dab, F(a)=F(b)\Leftrightarrow f(a)-\lambda g(a)=
f(b)-\lambda g(b)\Leftrightarrow\lambda=\frac{f(b)-f(a)}{g(b)-g(a)}\\[0.2cm]
\Rightarrow$ Rolle-tétel miatt $\exists\xi\in(a,b):F'(\xi)=0\Leftrightarrow F'(\xi)
=f'(\xi)-\lambda g'(\xi)=0\Leftrightarrow\lambda=\frac{f'(\xi)}{g'(\xi)}\bizva$
\\[0.2cm]\textbf{3. Lagrange-tétel}\\[0.1cm]\tetel Tfh. $f\cab,f\dab$\\[0.2cm]Ekkor
$\exists\xi\in(a,b):\frac{f(b)-f(a)}{b-a}=f'(\xi)$\\[0.2cm]\biz Legyen $g(x)=x
\Rightarrow g'(x)=1\neq0\quad$ Így alkalmazható rá a Cauchy-tétel$\bizva$
\newpage
\section{A monotonitásra vonatkozó elégséges, és \\ szükséges és elégséges feltételek.}
\subsection{Elégséges Feltétel}
\tetel Tfh. $f\dab$ Ekkor: \\[0.1cm]\hspace*{0.3cm}\textbf{i,} $f'\geq0\quad(a,b)$-n
$\Rightarrow f$ monoton nő $(a,b)$-n\\[0.1cm]\hspace*{0.3cm}\textbf{ii,}
$f'\leq0\quad(a,b)$-n $\Rightarrow f$ monoton fogy $(a,b)$-n\\[0.1cm]
\hspace*{0.3cm}\textbf{iii,} $f'>0\quad (a,b)$-n $\Rightarrow f$ szigorú monoton nő
$(a,b)$-n\\[0.1cm]\hspace*{0.3cm}\textbf{iv,} $f'<0\quad (a,b)$-n $\Rightarrow f$
szigorú monoton fogy $(a,b)$-n\\[0.2cm]\biz \\[0.1cm]\textbf{i,} Legyen
$[x_1,x_2]\subset(a,b)$ A Lagrange tétel miatt $\exists\xi\in
(x_1,x_2):\\[0.2cm]f(x_2)-f(x_1)=f'(\xi)(x_2-x_1)\geq0\Rightarrow f$ monoton
nő.\\[0.1cm]\textbf{ii,} Ugyanígy\\[0.1cm]\textbf{iii,} A Lagrange tétel után
$\Rightarrow\exists\xi\in (x_1,x_2):\\[0.2cm]f(x_2)-f(x_1)=f'(\xi)(x_2-x_1)>0
\Rightarrow f$ szigorú monoton nő\\[0.1cm]\textbf{iv,} Ugyanígy. $\bizva$
\subsection{Szükséges és elégséges feltétel}
\tetel Tfh. $f\dab$ Ekkor: \\[0.2cm]\textbf{i,} $f'\geq0\quad(a,b)$-n
$\Leftrightarrow f$ monoton nő $(a,b)$-n\\[0.2cm]\textbf{ii,} $f'\leq0\quad(a,b)$-n
$\Leftrightarrow f$ monoton fogy $(a,b)$-n\\[0.2cm]\textbf{iii,}
$f'\geq0\quad(a,b)$-n, de $\nexists(c,d)\subset(a,b):f'=0\quad(c,d)$-n
$\Leftrightarrow f$ szigorú monoton nő $(a,b)$-n\\[0.2cm]\textbf{iv,}
$f'\leq0\quad(a,b)$-n, de $\nexists(c,d)\subset(a,b):f'=0\quad(c,d)$-n
$\Leftrightarrow f$ szigorú monoton fogy $(a,b)$-n\\[0.2cm]\biz \textbf{i,}
"$\Rightarrow$"$\checkmark\quad$"$\Leftarrow$" Tfh. $f$ monoton nő és legyen
$\xi\in(a,b)$ tetszőleges\\[0.2cm] \[\displaystyle \frac{f(x)-f(\xi)}{x-\xi} = \left\{
\begin{gathered}
\geq0 \quad \text{, ha }x\geq\xi\hspace{5.7cm} \\
\geq0 \quad \text{, ha }x<\xi\hspace{5.7cm}
\end{gathered}\right. \]
\\[0.2cm]$\Rightarrow\exists f'(\xi)=\lim\limits_{x\to\xi}\frac{f(x)-f(\xi)}{x-\xi}$
\\[0.2cm]\textbf{ii,} Hasonló\\[0.2cm]\textbf{iii,} "$\Rightarrow$" $f'\geq0
\Rightarrow f$ szigorú monoton nő\\[0.2cm]Indirekten Tfh. $f$ nem szigorú monoton
\\[0.2cm]$\Rightarrow\exists c,d:f(c)=f(d)\Rightarrow f=f(c)\quad(c,d)$-n
$\Rightarrow f'=0\quad(c,d)$-n, ez ellentmondás\\[0.2cm]$\Rightarrow f$ szigorú
monoton nő.\\[0.2cm] "$\Leftarrow$" $f$ szigorú monoton nő $\Rightarrow f$ monoton
nő. $\Rightarrow f'\geq0$ Indirekten:\\[0.2cm] Tfh. $\exists(c,d)\subset(a,b):f'=0
\quad(c,d)$-n$\Rightarrow f$ konstans $(c,d)$-n $\Rightarrow f$ nem szigorú monoton
nő\\[0.2cm]És így ellentmondásra jutottunk $\Rightarrow\nexists(c,d):f'=0\quad
(c,d)\text{-n }$\\[0.2cm]\textbf{iv,} Hasonló $\bizva$
\newpage
\section{L'Hospital-szabályok}
\subsection{L'Hospital szabály $\frac{0}{0}$ alakra}
\tetel Tfh. \textbf{i,} $f,g\dab,\quad(-\infty\leq a<b<\infty)$\\[0.2cm]\hspace*{2cm}
\textbf{ii,} $g'(x)\neq0,\quad x\in(a,b)$\\[0.2cm]\hspace*{2.1cm}\textbf{iii,} $\limaj f=\limaj
g=0$\\[0.2cm]\hspace*{2.1cm}\textbf{iv,} $\exists\limaj \frac{f'}{g'}$ és
$\limaj\frac{f'}{g'}=A\in\Rv$\\[0.2cm] Ekkor: $\exists\limaj \frac{f}{g}$ és
$\limaj\frac{f}{g}=\limaj\frac{f'}{g'}$\\[0.2cm]\biz\textbf{i,} Tfh.
$a\neq-\infty$\\[0.2cm]Tudjuk: $\limaj\frac{f'}{g'}=A\Rightarrow\forall \varepsilon>0,
\exists x_0\in(a,b),\forall\xi\in(a,x_0):\frac{f'(\xi)}{g'(\xi)}
\in K_{\varepsilon}(A)$\\[0.2cm]Legyen $f(a)=g(a)=0$ és legyen $x\in(a,x_0)$
tetszőleges, ekkor $f,g\in C[a,x]$ és $f,g\in\D(a,x)$\\[0.2cm]$\Rightarrow$ a
Cauchy-középértéktétel miatt: $\exists\xi\in(a,x):\frac{f(x)-f(a)}{g(x)-g(a)}=
\frac{f'(\xi)}{g'(\xi)}\in K_{\varepsilon}(A)\\[0.2cm]\Rightarrow\forall\varepsilon >0,\exists
x_0\in(a,b),\forall x\in(a,x_0):\frac{f(x)}{g(x)}\in K_{\varepsilon}(A)
\Rightarrow\limaj\frac{f}{g}=A$\\[0.2cm]\textbf{ii,} Tfh. $a=-\infty\quad$
Visszavezetjük \textbf{i,}-re\\[0.1cm]Legyen $F(y):=f(b+1-\frac{1}{y}),\quad
y\in(0,1)$ és\\[0.1cm]$G(y):=g(b+1-\frac{1}{y}),\quad y\in(0,1)$\\[0.1cm]$y<1
\Rightarrow b+1-\frac{1}{y}<b\Rightarrow f\text{ és }g$ értelmezve van a
($b+1- \frac{1}{y}$) pontban.\\[0.1cm]$\limnj F=\lim\limits_{y\to0+0}f(b+1-
\frac{1}{y})=\lim\limits_{-\infty} f=0\\[0.2cm]\limnj G=\lim\limits_{-\infty}
g=0\quad$ Ha $\exists\limnj\frac{F}{G}$, ekkor\\[0.2cm]$
\limnj\frac{F}{G}=\lim\limits_{y\to0+0}\frac{f}{g}(b+1-\frac{1}{y})=
\lim\limits_{-\infty}\frac{f}{g}\\[0.2cm]F'(y)=f'(b+1-\frac{1}{y})\cdot
\frac{1}{y^2}\\[0.2cm]G'(y)=g'(b+1-\frac{1}{y})\cdot\frac{1}{y^2}
\neq0\quad\quad y\in(0,1)\\[0.2cm]\limnj\frac{F'}{G'}= \lim\limits_{-\infty}\frac{f'}{g'}\hspace{1.5cm}$ Alkalmazható \textbf{i,}
$F$ és $G$-re\\[0.2cm]$\Rightarrow\limnj\frac{F}{G}=\limnj\frac{F'}{G'}
\quad\Rightarrow\quad\lim\limits_{-\infty}\frac{f}{g}=\limnj\frac{F}{G} \text{ és }
\lim\limits_{-\infty}\frac{f'}{g'}=\limnj
\frac{F'}{G'}\bizva$\newpage
\subsection{L'Hospital szabály $\frac{\infty}{\infty}$ alakra}\tetel Tfh. \textbf{i,}
$f,g\dab,\quad(-\infty\leq a<b<\infty)$\\[0.2cm] \hspace*{2.1cm}\textbf{ii,}
$g'(x)\neq0,\quad x\in(a,b)$\\[0.2cm] \hspace*{2.1cm}\textbf{iii,} $\limaj f=\limaj g=\infty$
\\[0.2cm]\hspace*{2.1cm}\textbf{iv,} $\exists\limaj\frac{f'}{g'}$ és $\limaj
\frac{f'}{g'}=A\in\Rv$\\[0.2cm] Ekkor: $\exists\limaj\frac{f}{g}$ és $\limaj
\frac{f}{g}=\limaj\frac{f'}{g'}$\\[0.2cm]\biz\textbf{i,} Tfh.
$a\neq-\infty,A\in\R$\\[0.2cm]Tudjuk: $\limaj\frac{f'}{g'}=A\Rightarrow \forall\varepsilon>0,\exists x_0\in(a,b),
\forall\xi\in(a,x_0):\frac{f'(\xi)}{g'(\xi)}\in K_{\varepsilon}(A)$\\[0.1cm]Legyen $x\in(a,x_0)$
és alkalmazzuk a Cauchy középérték-tételt az $[x,x_0]$ intervallumra\\[0.1cm]
$\Rightarrow\exists\xi\in(x,x_0):\frac{f(x)-f(x_0)}{g(x)-g(x_0)}=
\frac{f'(\xi)}{g'(\xi)}$\hspace{1cm} Felthető, hogy $f>0\quad(a,x_0)$-n,
hiszen $\lim\limits_a f=\infty$\\[0.1cm]Hasonlóan $g>0\quad(a,x_0)$-n.\\[0.2cm]
$\frac{f(x)}{g(x)}\cdot\frac{1-\frac{f(x_0)}{f(x)}}{1-\frac{g(x_0)}
{g(x)}}=\frac{f'(\xi)}{g'(\xi)}\Rightarrow\frac{f(x)}{g(x)}=\frac{f'(\xi)}
{g'(\xi)}\cdot\underbrace{\frac{1-\frac{g(x_0)}{g(x)}}{1-\frac{f(x_0)}
{f(x)}}}_{T(x)}=\frac{f'(\xi)}{g'(\xi)}\cdot T(x)=\frac{f'(\xi)}{g'(\xi)}
\cdot(T(x)-1)+ \frac{f'(\xi)}{g'(\xi)}\\[0.2cm]\limaj T=1\Rightarrow\limaj(T-1)=0
\\[0.2cm]\frac{f'(\xi)}{g'(\xi)}\in K_{\varepsilon}
(A)\quad\Rightarrow\quad A-\varepsilon<\frac{f'(\xi)}{g'(\xi)}
<A+\varepsilon\quad\Rightarrow\quad \frac{f'(\xi)}{g'(\xi)}$ korlátos.
\\[0.2cm]$\Rightarrow\limaj\frac{f'(\xi)}{g'(\xi)}\cdot(T(x)-1)=0
\Rightarrow\forall\varepsilon>0,\exists x_1\in(a,x_0),\forall x\in(a,x_1):
\abs{\frac{f'(\xi)}{g'(\xi)}\cdot(T(x)-1)}<\varepsilon\\[0.2cm]
\frac{f(x)}{g(x)}-A=\frac{f'(\xi)}{g'(\xi)}\cdot(T(x)-1) +\frac{f'(\xi)}{g'(\xi)}-A
\\[0.2cm]\Rightarrow\abs{\frac{f(x)}{g(x)}-A}
\leq \underbrace{\abs{\frac{f'(\xi)}{g'(\xi)}\cdot(T(x)-1)}}_{<\varepsilon}
+\underbrace{\abs{\frac{f'(\xi)}{g'(\xi)}-A}}_{<\varepsilon}<2
\varepsilon\\[0.2cm]\Rightarrow\forall\varepsilon>0,\exists x_1\in(a,x_0),
\forall x\in(a,x_1):\abs{\frac{f(x)}{g(x)}-A}<2\varepsilon\Rightarrow\limaj \frac{f}{g}=A$
\\[0.2cm]\textbf{ii,} $a\neq-\infty,A=\infty\quad$
Láttuk:\\[0.1cm]$\frac{f(x)}{g(x)}=\frac{f'(x)}{g'(x)}\cdot
T(x)\\[0.2cm]\limaj T=1 \Rightarrow\exists x_1\in(a,x_0),\forall x\in(a,x_1):T(x)>\frac{1}{2}
\\[0.2cm]\frac{f'(x)}{g'(x)}\in K_{\varepsilon}
(\infty)\Rightarrow\frac{f'(x)}{g'(x)}>\frac{1}{\varepsilon}\Rightarrow
\frac{f(x)}{g(x)}>\frac{1}{2\varepsilon}\\[0.2cm]\forall\varepsilon>0
,\exists x_1\in(a,x_0),\forall x\in(a,x_1):\frac{f(x)}{g(x)}>\frac{1}
{2\varepsilon}\quad\Rightarrow\quad\limaj\frac{f}{g}=\infty=A$\\[0.2cm]
\textbf{iii,} $a\neq-\infty,A=-\infty\quad$ Hasonló \textbf{ii,}-hez\\[0.2cm]
\textbf{iv,} $a=-\infty$ Visszavezetjük az előzőre mint az előző tétel \textbf{ii,} részében. $\bizva$\newpage
\section{Taylor-formula a Lagrange-féle maradéktaggal.}
\tetel Ha $f\in\D^{(n+1)}(K(a))$, akkor \\[0.1cm] $\forall x\in K(a),\exists
\xi\in(a,x)\cup(x,a):f(x)=\sum\limits_{k=0}^{n}\frac{f^{(k)}(a)}{k!}
(x-a)^k+\frac{f^{(n+1)}(\xi)}{(n+1)!}(x-a)^{n+1}$\\[0.2cm]\biz
Legyen $F(x):=f(x)-\sum\limits_{k=0}^{n}\frac{f^{(k)}(a)}{k!}(x-a)^k\\[0.1cm]
F(a)=f(a)-f(a)=0\\[0.1cm]F'(x)=f'(x)-\sum\limits_{k=1}^{n}\frac{f^{(k)}(a)}{k!}
\cdot k\cdot(x-a)^{k-1}\\[0.1cm]\Rightarrow F'(a)=f'(a)-f'(a)=0\\[0.1cm]
F''(x)=f''(x)-\sum\limits_{k=2}^{n}\frac{f^{(k)}(a)}{k!}\cdot k\cdot(k-1)
(x-a)^{k-2}\\[0.2cm]F''(a)=f''(a)-f''(a)=0\quad\Rightarrow\quad F^{(n)}(a)=0,\quad
F^{(n+1)}(x)=f^{(n+1)}(x)$\\[0.2cm]Legyen $G(x)=(x-a)^{(n+1)}\Rightarrow G(a)=0
\\[0.2cm]G'(x)=(n+1)(x-a)^n\Rightarrow G'(a)=0,...,G''(a)=0\\[0.2cm]\Rightarrow
G^{(n)}(a)=0,\quad G^{(n+1)}(a)=(n+1)!$\\[0.2cm]Alkalmazzuk a Cauchy
középértéktételt:$\quad\exists$ ilyen $\xi_1,\xi_2...\xi_{n+1}\\[0.2cm]$
\[\frac{f(x)-\sum\limits_{k=0}^{n}\frac{f^{(k)}(a)}{k!}(x-a)^k}{(x-a)^{(n+1)}}=
\frac{F(x)}{G(x)}=\frac{F(x)-F(a)}{G(x)-G(a)}=\frac{F'(\xi_1)}{G'(\xi_1)}=
\frac{F'(\xi_1)-F'(a)}{G'(\xi_1)-G'(a)}=\frac{F''(\xi_2)}{G''(\xi_2)}=...=\] 
\[=\frac{F^{(n)}(\xi_n)}{G^{(n)}(\xi_n)}=\frac{F^{(n)}(\xi_n)-F^{(n)}(a)}
{G^{(n)}(\xi_n)-G^{(n)}(a)}=\frac{F^{(n+1)}(\xi_{n+1})}{G^{(n+1)}(\xi_{n+1})}=
\frac{f^{(n+1)}(\xi_{n+1})}{(n+1)!}\]
Legyen $\xi=\xi_{n+1}\bizva$
\subsection{Egy elégséges feltétel arra, hogy egy függvény Taylor-sora előállítsa a \\ függvényt.}
\tetel
Tfh. $f\in\D^\infty(K(a))\text{ és }sup\{\abs{f^{(n)}(x)}\quad n\in\N,x\in 
K(a)\}=M\text{ és }M<\infty$\\[0.2cm]Ekkor: $f(x)=\sum\limits_{k=0}^{\infty}\frac
{f^{(k)}(a)}{k!}(x-a)^k\quad(x\in K(a))$\\\biz
$\exists\xi\in(a,x):\abs{f-\sum\limits_{k=0}^{n}\frac{f^{(k)}(a)}{k!}(x-a)^k}=
\abs{\frac{f^{(n+1)}(\xi)}{(n+1)!}\cdot(x-a)^{n+1}}\leq\\[0.2cm]\leq
M\cdot\frac{\abs{x-a}^{n+1}}{(n+1)!}\to0\quad(n\to\infty)\quad\Rightarrow\quad
f(x)=\sum\limits_{k=0}^{\infty}\frac{f^{(k)}(a)}{k!}(x-a)^k\bizva$\newpage
\section{A konvexitásra és az inflexiós pontra vonatkozó szükséges és \\ elégséges feltételek többször differenciálható függvények esetében.}
\subsection{Konvexitásra}
\tetel $\fabr$\\[0.1cm]
\hspace*{0.5cm}\textbf{i,} Ha $f\dab$, akkor\\[0.1cm]
\hspace*{1cm}\textbf{a,} $f$ konvex $\Leftrightarrow f'\nearrow\quad(a,b)$-n
\\[0.1cm]\hspace*{1cm}\textbf{b,} $f$ szigorúan konvex $\Leftrightarrow f'
\uparrow\quad(a,b)$-n\\[0.2cm]
\hspace*{0.5cm}\textbf{ii,} Ha $f\in\D^2(a,b)$, akkor\\[0.1cm]
\hspace*{1cm}\textbf{a,} $f$ konvex $\Leftrightarrow f''\geq0\quad(a,b)$-n
\\[0.1cm]\hspace*{1cm}\textbf{b,} $f$ szigorúan konvex $\Leftarrow f''>0\quad(a,b)$-n\\[0.2cm]
\biz Elég \textbf{i,}-t bizonyítani\\[0.1cm]\hspace*{0.5cm}\textbf{a,}
"$\Rightarrow$" Tfh. $f$ konvex\\[0.1cm]Legyen $x_1<x_2$ tetszőleges és 
$x_1<y_1<y_2<x_2\\[0.2cm]\hspace*{1.3cm}\frac{f(y_1)-f(x_1)}{y_1-x_1}\quad
\leq\quad\frac{f(x_2)-f(x_1)}{x_2-x_1}\quad\leq\quad\frac{f(y_2)-f(x_2)}
{y_2-x_2}\\[0.2cm]\lim\limits_{y_1\to x_1+0}\frac{f(y_1)-f(x_1)}{y_1-x_1}=
f'(x_1)\hspace{0.6cm}\text{és}\hspace{0.4cm}\lim\limits_
{y_2\to x_2-0}\frac{f(y_2)-f(x_2)}{y_2-x_2}=f'(x_2)
\Rightarrow f'(x_1)\leq f'(x_2)\Rightarrow f'\nearrow$\\[0.2cm]
"$\Leftarrow$" Tfh. $f'\nearrow$ Elég:\\[0.2cm]
$\forall x_1,x_2\in(a,b),x_1<x_2,x\in(x_1,x_2):f(x)\leq\frac{f(x_2)-f(x_1)}
{x_2-x_1}\cdot(x-x_1)+f(x_1)$\\[0.1cm]Azaz 
$r(x):=f(x)-(\frac{f(x_2)-f(x_1)}{x_2-x_1}\cdot(x-x_1)+f(x_1))\leq0
\quad\Rightarrow\quad r(x_1)=0,\quad r(x_2)=0\\[0.2cm]\Rightarrow$
Rolle középértéktétel miatt: $\exists\xi\in(x_1,x_2):r'(\xi)=0\\[0.2cm]
r'(x)=f'(x)-\frac{f(x_2)-f(x_1)}{x_2-x_1}\quad\nearrow$\\[0.2cm]
$\Rightarrow r'\leq0\quad(x_1,\xi)\text{-n}\quad\Rightarrow r\searrow
\quad(x_1,\xi)\text{-n\quad és}\\[0.1cm]\hspace*{0.5cm}r'\geq0\quad(\xi,x_2)
\text{-n}\quad\Rightarrow r\nearrow\quad(\xi,x_2)$-n\\[0.2cm]
$\Rightarrow r\leq0\quad(x_1,x_2)\text{-n}$\\[0.2cm]
\hspace*{0.5cm}\textbf{b,} Hasonló $\bizva$
\subsection{Inflexiós pontra}
\tetel $\fabr,x_0\in(a,b)$\\[0.1cm]\textbf{i,} Ha $f$ kétszer folytonosan
deriválható és $x_0$ inflexiós pont, ekkor $f''(x_0)=0$\\[0.1cm]\textbf{ii,}
Ha $f$ háromszor folytonosan deriválható és $f''(x_0)=0$ és $f'''(x_0)\neq0$, ekkor
$x_0$ inflexiós pont.\\[0.2cm]\biz
\textbf{i,} Indirekten Tfh. $f''(x_0)\neq0$, pl: $f''(x_0)>0$\\[0.1cm]
$f''$ folytonos $\Rightarrow\exists K(x_0):f''>0\quad K(x_0)$-n\\[0.1cm]
Taylor formula\\[0.1cm] $n=1:\exists\xi:\underbrace{f(x)-(f(x_0)+f'(x_0)
(x-x_0))}_{l(x)}=\underbrace{\frac{f''(\xi)}{2!}(x-x_0)^2}_{\geq0}$\\[0.1cm]
$\Rightarrow l$ nem vált előjelet $\Rightarrow x_0$ nem inflexiós pont.\\[0.2cm]
\textbf{ii,} Tfh. $f'''(x_0)>0\Rightarrow\exists K(x_0):f'''>0\quad K(x_0)$-n
\\[0.1cm]Taylor formula $n=2:\\[0.1cm]\exists\xi:l(x)=f(x)-(f(x_0)+f'(x_0)(x-x_0)+
\frac{f''(x_0)}{2}(x-x_0)^2)=\underbrace{\frac{f'''(\xi)}{3!}}_{>0}
\cdot(x-x_0)^3\quad x\in K(x_0)$\\[0.2cm]A jobb oldal szigorúan előjelet
vált$\quad\Rightarrow\quad x_0$ inflexiós pont.$\bizva$\newpage
\section{A primitív függvény létezésére vonatkozó szükséges feltétel.}
\tetel Ha $I$ intervallum, és $f:\itr$ függvénynek $\exists$ \prfv e, akkor $f$ 
Darboux \\ tulajdonságú, azaz $\forall a,b\in I,a<b,\forall c\in(f(a),f(b)),\exists\xi
\in(a,b):f(\xi)=c$\\[0.2cm]\biz Tfh. $f(a)<f(b)$, legyen $f_1=f-c$, $f_1$-nek is
$\exists$ \prfv e, mégpedig\\[0.2cm]$F_1(x)=F(x)-cx$, ahol $F$ az $f$ \prfv e,
hiszen $F_1'(x)=F'(x)-c=f(x)-c=f_1(x)$\\[0.2cm]Ekkor:
$F_1'(a)=f_1(a)=f(a)-c<0\\[0.2cm]\hspace*{1.3cm}F_1'(b)=f_1(b)=f(b)-c>0\\[0.2cm]
\Rightarrow F_1'(a)=\lim\limits_{x\to a+0}\frac{F_1(x)-F_1(a)}{x-a}=f_1(a)<0
\quad\Rightarrow\quad\exists\delta>0,\forall x\in(a,a+\delta):
\frac{F_1(x)-F_1(a)}{x-a}<0$\\[0.2cm]itt $x-a>0\Rightarrow\exists\delta>0,
\forall x\in(a,a+\delta):F_1(x)<F_1(a)\\[0.2cm]F_1'(b)=\lim
\limits_{x\to b-0}\frac{F_1(x)-F_1(b)}{x-b}=f_1(b)>0\quad\Rightarrow\quad
\exists\delta>0,\forall x\in(b-\delta,b):\frac{F_1(x)-F_1(b)}{x-b}>0
\\[0.2cm]x-b<0\Rightarrow\exists\delta>0,\forall x\in(b-\delta,b):
F_1(x)<F_1(b)\quad\Rightarrow F_1\in\D(I)\Rightarrow F_1\in C[a,b]$\\[0.2cm]
A Weierstrass-tétel miatt $F_1$-nek $\exists$ abszolút minimuma, azaz
$\exists\xi\in[a,b]:F_1(\xi)=\min\limits_{[a,b]}F_1\\[0.2cm]
\xi\neq a,\xi\neq b\Rightarrow\xi\in(a,b)\Rightarrow\xi$-ben lokális minimum
$\Rightarrow F_1'(\xi)=0\Rightarrow f_1(\xi)=f(\xi)-c=0\bizva$
\section{Az integrálhatóság jellemzése az oszcillációs összegekkel.}
\tetel
$f\in\rab\Leftrightarrow\forall\varepsilon>0,\exists\tau\in\fab:\Omega\ftau<
\varepsilon$\\[0.2cm]
\biz "$\Leftarrow$" Tfh. $\varepsilon$-hoz $\exists\tau:\Omega\ftau<\varepsilon
\\[0.2cm]I^*f-I_*f<\varepsilon,\quad\varepsilon$ tetszőleges\\[0.2cm]
$\Rightarrow I^*f=I_*f$\\[0.2cm]
"$\Rightarrow$" Tfh. $f\in\rab\Rightarrow\forall\varepsilon>0,\exists\te\in\fab
:If-\frac{\varepsilon}{2}<s(f,\te)\leq If$\\[0.2cm]Hasonlóan:
$\exists\tk\in\fab:If\leq S(f,\tk)<If+\frac{\varepsilon}{2}$\\[0.2cm]
Legyen $\tau=\te\cup\tk\Rightarrow\\[0.2cm]If-\frac{\varepsilon}{2}<s(f,\te)\leq
\underline{s\ftau}\leq If\leq \underline{S\ftau}\leq S(f,\tk)<If
+\frac{\varepsilon}{2}\\[0.3cm]\Rightarrow S\ftau-s\ftau<\varepsilon\bizva$
\section{Az integrálhatóság jellemzése alsó és felső közelítő összegek \\ határértékével}
\tetel\\[0.2cm]$f\in\rab$ és $\int\limits_{a}^{b}f=I\Leftrightarrow
\exists\tau_n:\lim s(f,\tau_n)=\lim S(f,\tau_n)=I$\\[0.2cm]\biz
"$\Rightarrow$" Tfh. $f\in\rab\Rightarrow\forall\varepsilon>0,\exists\tau\in\fab:
If-\frac{\varepsilon}{2}<s\ftau\leq S\ftau<If+\frac{\varepsilon}{2}$
\\[0.2cm]Legyen $\frac{\varepsilon}{2}=\frac{1}{n},\quad\tau=\tau_n\\[0.2cm]
\underbrace{I-\frac{1}{n}}_{\to I}\leq\underbrace{s(f,\tau_n)}_{\to I}\leq
\underbrace{S(f,\tau_n)}_{\to I}<\underbrace{I+\frac{1}{n}}_{\to I}$
\\[0.2cm]"$\Leftarrow$" Tfh. $\lim s(f,\tau_n)=\lim S(f,\tau_n)=I\Rightarrow
I_*f=I^*f=I\bizva$\newpage
\section{Műveletek integrálható függvényekkel}
\tetel Tfh. $f,g\in\rab$ Ekkor:\\[0.2cm]
\textbf{i,} $f+g\in\rab$ és $\intab f+g=\intab f+\intab g$\\[0.2cm]
\textbf{ii,} $\lambda\cdot f\in\rab$ és $\intab\lambda f=\lambda\cdot\intab f
\quad\lambda\in\R$\\[0.2cm]
\textbf{iii,} $f\cdot g\in\rab$\\[0.2cm]
\textbf{iv,} Ha $\abs{g(x)}\geq m>0\quad\forall x\in[a,b]$, akkor $\frac{f}{g}\in\rab$\\[0.2cm]
\biz Legyen $\tau=\{x_0,x_1,...,x_n\}\in\fab,\\[0.2cm]F_i:=\sup\limits_{\intv}f,
\quad f_i:=\inf\limits_{\intv}f\quad G_i:=\sup\limits_{\intv}g\quad
g_i:=\inf\limits_{\intv}g$\\[0.3cm]
\textbf{i,}$\quad f_i+g_i\leq f(x)+g(x)\leq F_i+G_i,\quad x\in\intv\\[0.2cm]
\Rightarrow f_i+g_i\leq\inf\limits_{\intv}(f+g)\leq\sup\limits_{\intv}(f+g)\leq
F_i+G_i\quad\quad/\cdot(x_i-x_{i-1})\\[0.2cm]\Rightarrow
s\ftau+s(g,\tau)\leq s(f+g,\tau)\leq S(f+g,\tau)\leq S\ftau+S(g,\tau)$\\[0.2cm]
Legyen $\te,\tk\in\fab$ tetszőleges és $\tau=\te\cup\tk\\[0.2cm]\Rightarrow
s(f,\te)+s(g,\tk)\leq s\ftau+s(g,\tau)\leq s(f+g,\tau)\leq I_*(f+g)\leq I^*
(f+g)\leq S(f+g,\tau)\leq\\[0.2cm]\leq S\ftau+S(g,\tau)\leq S(f,\te)+S(g,\tk)
\quad/\cdot\sup\limits_{\te},\inf\limits_{\te},\sup\limits_{\tk},\inf\limits_{\tk}
\\[0.2cm]\Rightarrow I_*(f)+I_*(g)\leq I_*(f+g)\leq I^*(f+g)\leq I^*(f)+I^*(g),
\quad\text{Mivel }I_*(f)=I^*(f)$ (ugyanez $g$-re)\\[0.2cm]
$\Rightarrow I_*(f+g)=I^*(f+g)\text{ és }\intab f+g=\intab f+\intab g$\\[0.2cm]
\textbf{ii,} Tfh. $\lambda\geq0\Rightarrow s(\lambda f,\tau)=\lambda\cdot s\ftau
\quad\quad(\inf\limits_{\intv}\lambda f=\lambda\cdot\inf\limits_{\intv}f)\\[0.2cm]
\Rightarrow I_*(\lambda f)=\lambda\cdot I_*(f)$\hspace{1cm}Hasonlóan:
$S(\lambda f,\tau)=\lambda\cdot S\ftau\Rightarrow I^*(\lambda f)=
\lambda\cdot I^*(f)\\[0.1cm]\Rightarrow I_*(\lambda f)=I^*(\lambda f)\text{ és }
\intab\lambda f=\lambda\cdot\intab f$\\[0.1cm]Tfh. $\lambda<0\\[0.1cm]
s(\lambda f,\tau)=\lambda\cdot S\ftau\Rightarrow I_*(\lambda f)=\lambda\cdot I^*(f)
\quad\text{és}\quad S(\lambda f,\tau)=\lambda\cdot s\ftau\Rightarrow I^*(\lambda f)=
\lambda\cdot I_*(f)\\[0.2cm]\Rightarrow I_*(\lambda f)=I^*(\lambda f)\text{ és }
\intab\lambda f=\lambda\cdot\intab f$\\[0.2cm]\textbf{iii,}
Oszcillációs összeggel: Tfh. $f,g\geq0\quad[a,b]$-n\\[0.1cm]
$f_i\cdot g_i\leq f(x)\cdot g(x)\leq F_i\cdot G_i\quad x\in\intv\\[0.2cm]
\Rightarrow f_i\cdot g_i\leq\inf\limits_{\intv}(f\cdot g)\leq
\sup\limits_{\intv}(f\cdot g)\leq F_i\cdot G_i\\[0.1cm]\Omega(f\cdot g,\tau)=
\sumi(\sup\limits_{\intv}(f\cdot g)-\inf\limits_{\intv}(f\cdot g))\cdot
(x_i-x_{i-1})\leq\sumi(F_i\cdot G_i-f_i\cdot g_i)\cdot(x_i-x_{i-1})=\\[0.1cm]=
\sumi(F_i\cdot G_i-F_i\cdot g_i+F_i\cdot g_i-f_i\cdot g_i)\cdot(x_i-x_{i-1})=
\\[0.1cm]=\sumi F_i(G_i-g_i)\cdot(x_i-x_{i-1})+\sumi g_i(F_i-f_i)\cdot(x_i-x_{i-1})
\\[0.2cm]f\in\rab\Rightarrow f$ korlátos $\Rightarrow F_i\leq M$ és $g_i\leq M\quad
\forall i=1,...,n$\newpage
$\Rightarrow\Omega(f\cdot g,\tau)\leq M\cdot\Omega(g,\tau)+M\cdot\Omega\ftau\\[0.2cm]
\Rightarrow\forall\varepsilon>0,\exists\te:\Omega(g,\te)<\varepsilon\quad\text{és}
\quad\forall\varepsilon>0,\exists\tk:\Omega(f,\tk)<\varepsilon$\\[0.2cm]Legyen
$\tau=\te\cup\tk\Rightarrow\Omega(g,\tau)\leq\Omega(g,\te)<\varepsilon\quad$
Hasonlóan: $\Omega\ftau\leq\Omega(f,\tk)<\varepsilon\\[0.2cm]\Rightarrow
\Omega(f\cdot g,\tau)<2\varepsilon M\Rightarrow f\cdot g\in\rab$\\[0.2cm]
Ha $f$ és $g$ tetszőleges, akkor legyen $m_f:=\inf\limits_{[a,b]}f,\quad
m_g:=\inf\limits_{[a,b]}g\Rightarrow\underbrace{f-m_f}_{\in\rab}\geq0,
\quad\underbrace{g-m_g}_{\in\rab}\geq0\\[0.2cm]\Rightarrow\underbrace{(f-m_f)(g-m_g)}
_{\in\rab}=f\cdot g\underbrace{-g\cdot m_f-f\cdot m_g+m_f\cdot m_g}_{\in\rab}
\Rightarrow f\cdot g\in\rab$\\[0.2cm]\textbf{iv,} Elég: $\frac{1}{g}\in\rab$\\[0.2cm]
\[\frac{1}{g(x)}-\frac{1}{g(y)}=\frac{g(y)-g(x)}{g(x)\cdot g(y)}\leq
\frac{\abs{g(y)-g(x)}}{\abs{g(x)\cdot g(y)}}\leq\frac{G_i-g_i}{m^2}\Rightarrow
\sup\limits_{\intv}\frac{1}{g}-\inf\limits_{\intv}\frac{1}{g}\leq
\frac{G_i-g_i}{m^2}\]
$\Rightarrow\Omega(\frac{1}{g},\tau)=\sumi(\sup\limits_{\intv}\frac{1}{g}-
\inf\limits_{\intv}\frac{1}{g})\cdot(x_i-x_{i-1})\leq
\frac{1}{m^2}\Omega(g,\tau)\\[0.2cm]\forall\varepsilon>0,\exists\tau,\Omega(g,\tau)
<\varepsilon\Rightarrow\Omega(\frac{1}{g},\tau)\leq\frac{\varepsilon}{m^2}\bizva$
\section{Folytonos függvény integrálható}
\tetel Ha $f\cab$, ekkor $f\in\rab$\\[0.2cm]
\biz Ha $f\cab\Rightarrow$ Heine tétel miatt $f$ egyenletesen folytonos, azaz\\[0.2cm]
$\forall\varepsilon>0,\exists\delta>0,\forall x,y\in[a,b],\abs{x-y}<\delta:
\abs{f(x)-f(y)}<\varepsilon$\\[0.2cm]
Legyen $\tau\in\fab$ olyan, hogy $\abs{\tau}<\delta\\[0.2cm]
\Omega\ftau=\sumi(\sup\limits_{\intv}f-\inf\limits_{\intv}f)(x_i-x_{i-1})=\sumi
\underbrace{\sup\limits_{x,y\in\intv}\abs{f(x)-f(y)}}_{<\varepsilon}\cdot(x_i-x_{i-1})
\leq\varepsilon\cdot(b-a)\\[0.2cm]\Rightarrow f\in\rab\bizva$
\section{Monoton függvény integrálható}
\tetel $f:[a,b]\to\R$ monoton, ekkor $f\in\rab$\\[0.2cm]
\biz Tfh. $f\nearrow$\\[0.2cm]$\Omega\ftau=\sumi
(\sup\limits_{\intv}f-\inf\limits_{\intv}f)\cdot(x_i-x_{i-1})=\sumi
(f(x_i)-f(x_{i-1}))\cdot(x_i-x_{i-1})$\\[0.2cm]Tfh. $\abs{\tau}<\delta\Rightarrow
\Omega\ftau\leq\delta\cdot\sumi(f(x_i)-f(x_{i-1}))=\delta\cdot(f(b)-f(a))<\varepsilon$
\\[0.2cm]Ha a $\delta<\frac{\varepsilon}{f(b)-f(a)}\Rightarrow f\in\rab\bizva$
\newpage
\section{Newton-Leibniz-tétel}
\tetel Ha $f\in\rab$ és $f$-nek $\exists F$ \prfv e, akkor: $\intab f=F(b)-F(a)$\\[0.1cm]
\biz Legyen $\tau=\{a=x_0<x_1<...<x_n=b\}\in\fab\\[0.2cm]\Rightarrow F(b)-F(a)=
F(x_n)-F(x_0)=F(x_n)-F(x_{n-1})+F(x_{n-1})-F(x_{n-2})+...+F(x_1)-F(x_0)=\\[0.2cm]=
\sumi(F(x_i)-F(x_{i-1}))$ Alkalmazzuk a Lagrange középértéktételt az $\intv$
intervallumon\\[0.2cm]$\exists\xi_i\in\intv:F(x_i)-F(x_{i-1})=F'(\xi_i)
(x_i-x_{i-1})=f(\xi_i)(x_i-x_{i-1})\\[0.2cm]\Rightarrow
s\ftau\leq F(b)-F(a)=\sumi(f(\xi_i)\cdot(x_i-x_{i-1}))\leq S\ftau\quad$
/sup a bal oldalon és inf a jobb oldalon\\
$\Rightarrow I_*f\leq F(b)-F(a)\leq I^*f\quad\quad$
Mivel $I_*f=I^*f=\intab f\Rightarrow F(b)-F(a)=\intab f\bizva$
\section{A differenciál- és integrálszámítás alaptétele}
\tetel Legyen $f\in\rab,x_0\in[a,b],F(x)=\int\limits_{x_0}^xf\quad(x\in[a,b])$, ekkor:
\\[0.2cm]\textbf{i,} $F\cab$\\[0.2cm]
\textbf{ii,} Ha $f\in C(d)$, akkor $F\in D(d)$ és $F'(d)=f(d)\quad(d\in[a,b])$\\[0.2cm]
\biz \textbf{i,} $f\in\rab\Rightarrow f$ korlátos $\Rightarrow\exists M:\abs{f}\leq M
\\[0.2cm]\abs{F(x_2)-F(x_1)}=\abs{\int\limits_{x_0}^{x_2}f-\int\limits_{x_0}^{x_1}f}=
\abs{\int\limits_{x_1}^{x_2}f}\leq\abs{\int\limits_{x_1}^{x_2}\abs{f}}\leq M\cdot
\abs{x_2-x_1}\Rightarrow x_2\to x_1\Rightarrow F(x_2)\to F(x_1)\\[0.2cm]
\Rightarrow F\in C(x_1)\quad x_1$ tetszőleges\\[0.2cm]
\textbf{ii,} Igazolni kell, hogy $f(d)=F'(d)=\limh\frac{F(d+h)-F(d)}{h}$, azaz
$\limh\abs{\frac{F(d+h)-F(d)}{h}-f(d)}=0\\[0.2cm]
\abs{\frac{F(d+h)-F(d)}{h}-f(d)}=\abs{\frac{1}{h}\cdot\int\limits_d^{d+h}f(t)dt-f(d)}=
\abs{\frac{1}{h}\cdot\int\limits_d^{d+h}f(t)-f(d)dt}\leq\frac{1}{h}\cdot
\int\limits_d^{d+h}\abs{f(t)-f(d)}dt\\[0.2cm]
f\in C(d)\Rightarrow\forall\varepsilon>0,\exists\delta>0,\forall t\in[a,b],
\abs{t-d}<\delta:\abs{f(t)-f(d)}<\varepsilon$\\[0.3cm]Legyen
$\abs{h}<\delta\Rightarrow\abs{t-d}\leq\abs{h}<\delta\Rightarrow\forall\varepsilon>0,
\exists\delta>0,\forall\abs{h}<\delta:\abs{\frac{F(d+h)-F(d)}{h}-f(d)}<\varepsilon
\\[0.2cm]\Rightarrow\limh\abs{\frac{F(d+h)-F(d)}{h}-f(d)}=0\bizva$
\end{document}